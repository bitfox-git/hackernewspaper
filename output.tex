\documentclass[10pt,a4paper]{article} 
\usepackage[a4paper, margin=1.5cm, tmargin=2.0cm, bmargin=2.0cm]{geometry}

\frenchspacing 

\usepackage{graphicx} 
\usepackage{amssymb,amsmath} 
\usepackage{multicol} 
\usepackage{url}
\usepackage{wrapfig} 
\usepackage[T1]{fontenc} 
\usepackage[pdfpagemode=FullScreen, colorlinks=false]{hyperref} 
\usepackage{fancyhdr} 

% https://mirrors.ibiblio.org/CTAN/fonts/fontawesome/doc/fontawesome.pdf
\usepackage{fontawesome}
\pagestyle{fancy} 
\usepackage{xcolor}
\usepackage{titlesec}


% Headers
\lhead{hacker{\color{red}news}paper }
\rhead{\rightmark }
% Footers
\lfoot{\footnotesize 
\href{https://www.bitfox.nl}{\includegraphics[height=1em]{bflogo.jpg}} \quad 
\faGlobe \href{https://hackernewsletter.com/}{hackernewsletter.com} 
}
\cfoot{} 
\rfoot{\footnotesize Page \thepage}

% Lines
\renewcommand{\headrulewidth}{0.4pt} 
\renewcommand{\footrulewidth}{0.4pt} 

% Metadata
\hypersetup{
    pdftitle = {HackerNewPaper 667},
    pdfauthor = {Bitfox},
}


\titleformat{\section}
{\normalfont\Huge\scshape\color{red}}
{\noindent}
{0em}{}

\titleformat{\subsection}
{\normalfont\large\bfseries}
{\noindent}
{0em}{}

\usepackage{fontspec}
\setmainfont{texgyrepagella}[
  Extension = .otf,
  UprightFont = *-regular,
  BoldFont = *-bold,
  ItalicFont = *-italic,
  BoldItalicFont = *-bolditalic,
]

\begin{document}
\thispagestyle{empty}
% put LOGO 
\Huge \usefont{T1}{phv}{b}{n} 
\noindent\textbf{hacker{\color{red}news}paper}
\normalfont
\normalsize
\hfill Issue \#667

{\noindent\color{red} \rule{\linewidth}{0.5mm}}

\begin{quotation}
    \textit{
Some people dream of success, while other people get up every morning and make it happen. } \par\hfill --- Wayne Huizenga

 
\end{quotation}

\tableofcontents
\vfill
\noindent\color{gray}
\huge \textsc{\_\_End\_\_ } \\ \\ 
\small Hackernewsletter is published by \href{www.curpress.com}{Curpress} from Bellingham, Washington. The hackernewspaper is an idea of \href{www.bitfox.nl}{Bitfox} and derives it's contents directly from the hackernewspaper. If you like this hackernewspaper, please subscribe to the original hackernewsletter at \href{www.hackersnewsletter.com}{www.hackersnewsletter.com}! Hacker Newsletter and Hacker Newspaper are not affiliated with Y Combinator in any way.

\color{black}

\newpage
\pagestyle{fancy}

\section{\#Favorites}

\subsection{OpenTF announces fork of Terraform}
\noindent\begin{minipage}[t]{0.19\linewidth}
\vspace{0pt}
\noindent\scshape\footnotesize
\\ {\scriptsize\faCalendar}\space 
2022-01-01
\\ {\scriptsize\faGlobe}\space 
nudgesecurity.com
\\ {\scriptsize\faThumbsOUp}\space 
\href{http://news.ycombinator.com/item?id=37262440\&utm\_term=comment}{1707} 
\\ {\scriptsize\faComments}\space 
\href{http://news.ycombinator.com/item?id=37262440\&utm\_term=comment}{80} 
\end{minipage} 
\begin{minipage}[t]{0.80\linewidth}
\vspace{0pt}
\begin{multicols}{2}
    \href{https://opentf.org/announcement?utm\_source=hackernewsletter\&utm\_medium=email\&utm\_term=fav}{
        \includegraphics[width=0.99\linewidth]{0.png}
    }
\paragraph{SaaS and cloud apps introduced outside of IT
Discover what technology is actually being used across your organization and who first adopted it.}
 Nudge Security inventories and auto-categorizes all cloud-delivered services: IaaS, PaaS, and SaaS.
Identities and accounts
Discover all cloud and SaaS accounts, users, and authentication methods as they are created—including the ones network and endpoint security controls miss.
Deep context
Deepen your understanding of cloud and SaaS use with visibility into resources and assets within cloud services, like files, repos, Slack channels, domains, and even billing information.
Historical use
Look back to understand who first onboarded a SaaS application, and who might still have unauthorized SaaS access to forgotten or abandoned accounts.
OAuth grants
Understand how data is shared across SaaS applications using OAuth grants, quickly surface overly permissive scopes, and easily revoke OAuth grants for Google Workspace and Microsoft 365 that employees no longer need.
\dots
\end{multicols}
\end{minipage}
\par\medskip
%\noindent\textcolor{red}{\rule{\linewidth}{0.2mm}}
\subsection{Factorio: Space Age}
\noindent\begin{minipage}[t]{0.19\linewidth}
\vspace{0pt}
\noindent\scshape\footnotesize
\\ {\scriptsize\faUser}\space 
Kovarex
\\ {\scriptsize\faCalendar}\space 
2023-08-25
\\ {\scriptsize\faThumbsOUp}\space 
\href{http://news.ycombinator.com/item?id=37260637\&utm\_term=comment}{1438} 
\\ {\scriptsize\faComments}\space 
\href{http://news.ycombinator.com/item?id=37260637\&utm\_term=comment}{53} 
\end{minipage} 
\begin{minipage}[t]{0.80\linewidth}
\vspace{0pt}
\begin{multicols}{2}
    \href{https://factorio.com/blog/post/fff-373?utm\_source=hackernewsletter\&utm\_medium=email\&utm\_term=fav}{
        \includegraphics[width=0.99\linewidth]{1.png}
    }
\paragraph{Hello, long time no see!
Today we are going to talk about the expansion which is called Factorio: Space Age.}

(Click here for static image version)
Factorio: Space Age continues the player's journey after launching rockets into space. Discover new worlds with unique challenges, exploit their novel resources for advanced technological gains, and manage your fleet of interplanetary space platforms.
Vanilla Factorio ends by launching the rocket into space, so I always expected it to be quite obvious what we are going to do next. Honestly the space platform related gameplay was actually planned a very long time ago, and we had shown a little bit in FFF-74. We had a similar kind of story with the Spidertron. Its concept was teased for the first time in FFF-120, just to be quietly abandoned and then revived 4 years later.
But as you might or might not know, the name 'WUBE' is an abbreviation of Wszystko będzie, which means something like "Everything will be done eventually". We didn't abandon the plan, we just realized it would be way too ambitious to try to fit it into the 1.0 Factorio rel
\dots
\end{multicols}
\end{minipage}
\par\medskip
%\noindent\textcolor{red}{\rule{\linewidth}{0.2mm}}
\subsection{ISPs should not police online speech no matter how awful it is}
\noindent\begin{minipage}[t]{0.19\linewidth}
\vspace{0pt}
\noindent\scshape\footnotesize
\\ {\scriptsize\faUser}\space 
Electronic Frontier Foundation
\\ {\scriptsize\faCalendar}\space 
2023-08-29
\\ {\scriptsize\faGlobe}\space 
eff.org
\\ {\scriptsize\faThumbsOUp}\space 
\href{http://news.ycombinator.com/item?id=37313349\&utm\_term=comment}{1434} 
\\ {\scriptsize\faComments}\space 
\href{http://news.ycombinator.com/item?id=37313349\&utm\_term=comment}{99} 
\end{minipage} 
\begin{minipage}[t]{0.80\linewidth}
\vspace{0pt}
\begin{multicols}{2}
    \href{https://www.eff.org/deeplinks/2023/08/isps-should-not-police-online-speech-no-matter-how-awful-it?utm\_source=hackernewsletter\&utm\_medium=email\&utm\_term=fav}{
        \includegraphics[width=0.99\linewidth]{2.png}
    }
\paragraph{Entrusting our speech to multiple different corporate actors is always risky.}
 Yet given how most of the internet is currently structured, our online expression largely depends on a set of private companies ranging from our direct Internet service providers and platforms, to upstream ISPs (sometimes called Tier 2 and 3), all the way up to Tier 1 ISPs (or the Internet backbone) that have no direct relationships with most users.
Tier 1 ISPs play a unique role in the internet “stack,” because numerous other service providers depend on Tier 1 companies to serve their customers. As a result, Tier 1 providers can be especially powerful chokepoints—given their reach, their content policies can affect large swaths of the web. At the same time given their distant relationship to speakers, Tier 1 ISPs have little if any context to make good decisions about their speech.
At EFF, we have long represented and assisted people from around the world—and across various political spectrums—facing censorship. That experience tells us that one of the most dangerous types of censorship happens at the site
\dots
\end{multicols}
\end{minipage}
\par\medskip
%\noindent\textcolor{red}{\rule{\linewidth}{0.2mm}}
\subsection{ChatGPT Enterprise}
\noindent\begin{minipage}[t]{0.19\linewidth}
\vspace{0pt}
\noindent\scshape\footnotesize
\\ {\scriptsize\faUser}\space 
Authors
\\ {\scriptsize\faCalendar}\space 
2023-08-28
\\ {\scriptsize\faThumbsOUp}\space 
\href{http://news.ycombinator.com/item?id=37297304\&utm\_term=comment}{858} 
\\ {\scriptsize\faComments}\space 
\href{http://news.ycombinator.com/item?id=37297304\&utm\_term=comment}{62} 
\end{minipage} 
\begin{minipage}[t]{0.80\linewidth}
\vspace{0pt}
\begin{multicols}{2}
    \href{https://openai.com/blog/introducing-chatgpt-enterprise?utm\_source=hackernewsletter\&utm\_medium=email\&utm\_term=fav}{
        \includegraphics[width=0.99\linewidth]{3.png}
    }
We’re launching ChatGPT Enterprise, which offers enterprise-grade security and privacy, unlimited higher-speed GPT-4 access, longer context windows for processing longer inputs, advanced data analysis capabilities, customization options, and much more. We believe AI can assist and elevate every aspect of our working lives and make teams more creative and productive. Today marks another step towards an AI assistant for work that helps with any task, is customized for your organization, and that protects your company data.
We’ve seen unprecedented demand for ChatGPT inside organizations
Since ChatGPT's launch just nine months ago, we’ve seen teams adopt it in over 80 of Fortune 500 companies.[\^footnote-1] We've heard from business leaders that they’d like a simple and safe way of deploying it in their organization. Early users of ChatGPT Enterprise—industry leaders like Block, Canva, Carlyle, The Estée Lauder Companies, PwC, and Zapier—are redefining how they operate and are using ChatGPT to craft clearer communications, accelerate coding tasks, rapidly explore answers to complex busi
\dots
\end{multicols}
\end{minipage}
\par\medskip
%\noindent\textcolor{red}{\rule{\linewidth}{0.2mm}}
\subsection{E-ink is so Retropunk}
\noindent\begin{minipage}[t]{0.19\linewidth}
\vspace{0pt}
\noindent\scshape\footnotesize
\\ {\scriptsize\faCalendar}\space 
2023-08-01
\\ {\scriptsize\faThumbsOUp}\space 
\href{http://news.ycombinator.com/item?id=37272652\&utm\_term=comment}{823} 
\\ {\scriptsize\faComments}\space 
\href{http://news.ycombinator.com/item?id=37272652\&utm\_term=comment}{69} 
\end{minipage} 
\begin{minipage}[t]{0.80\linewidth}
\vspace{0pt}
\begin{multicols}{2}
    \href{https://rmkit.dev/eink-is-so-retropunk/?utm\_source=hackernewsletter\&utm\_medium=email\&utm\_term=fav}{
        \includegraphics[width=0.99\linewidth]{4.png}
    }
\paragraph{E-ink is so Retropunk
I’ve long been struggling to describe why e-ink is so much fun for me, but I think I’ve finally realized what it is: e-ink is just so so so retropunk.}

An e-ink device is a hacker’s dream - or at least, this hacker’s dream. They
are a return to the magical feeling of computers of the 80s and 90s. It’s a
world where
Microsoft Windows and Apple Mac OS X never existed and we
don’t have to suffer with abstractions on top of abstractions. It’s DOS for
the 2020s. It’s graphing calculators for grown-ups.
Features
The e-ink devices I favor are low powered ARM devices running linux without a display server or gigabytes of RAM. Let’s break down why that’s so awesome:
- e-ink: The display is visible in the light of day :-D
- Low Powered: They can last for weeks on a single charge
- ARM is a simple architecture with a low instruction count and even lower cost
- Linux: as much as I like it, Android is a complicated mess
- Apps are simple: they talk to the kernel to read input and draw directly to the framebuffer
- Low RAM and slow CPUs: There’s no room to build complicated st
\dots
\end{multicols}
\end{minipage}
\par\medskip
%\noindent\textcolor{red}{\rule{\linewidth}{0.2mm}}
\subsection{Absurd Success}
\noindent\begin{minipage}[t]{0.19\linewidth}
\vspace{0pt}
\noindent\scshape\footnotesize
\\ {\scriptsize\faCalendar}\space 
2023-08-30
\\ {\scriptsize\faGlobe}\space 
marginalia.nu
\\ {\scriptsize\faThumbsOUp}\space 
\href{http://news.ycombinator.com/item?id=37331778\&utm\_term=comment}{587} 
\\ {\scriptsize\faComments}\space 
\href{http://news.ycombinator.com/item?id=37331778\&utm\_term=comment}{19} 
\end{minipage} 
\begin{minipage}[t]{0.80\linewidth}
\vspace{0pt}
\begin{multicols}{2}
    \href{https://www.marginalia.nu/log/87\_absurd\_success/?utm\_source=hackernewsletter\&utm\_medium=email\&utm\_term=fav}{
        \includegraphics[width=0.99\linewidth]{5.png}
    }
\paragraph{So… I’ve had the most unreal week of coding.}
 Zero exaggeration, I’ve halved the RAM requirements of the search engine, removed the need to take the system offline during an upgrade, removed hard limits on how many documents can be indexed, and quadrupled soft limits on how many keywords can be in the corpus.
It’s been a long term goal to keep it possible to run and operate the system on low-powered hardware, and so far improvements have been made, to the point where my 32 Gb RAM developer machine feels spacey rather than cramped, but this set of changes takes it several notches further.
But to roll back the tape to more somber times.
Marginalia Search went offline for almost a week due to some problems with the latest release.
I won’t go into the details too much here, it was a string of fairly trivial scaling problems in a process with a run time of 1 day when it works well and 2-3 days when it doesn’t. Had to restart it a few times, mostly because it ran out of RAM.
The cause of the outage was the fact that the system needs to go offline during an index switch in the first place. I
\dots
\end{multicols}
\end{minipage}
\par\medskip
%\noindent\textcolor{red}{\rule{\linewidth}{0.2mm}}
\subsection{CLI text processing with GNU awk}
\noindent\begin{minipage}[t]{0.19\linewidth}
\vspace{0pt}
\noindent\scshape\footnotesize
\\ {\scriptsize\faGlobe}\space 
github.io
\\ {\scriptsize\faThumbsOUp}\space 
\href{http://news.ycombinator.com/item?id=37290356\&utm\_term=comment}{418} 
\\ {\scriptsize\faComments}\space 
\href{http://news.ycombinator.com/item?id=37290356\&utm\_term=comment}{21} 
\end{minipage} 
\begin{minipage}[t]{0.80\linewidth}
\vspace{0pt}
\begin{multicols}{2}
    \href{https://learnbyexample.github.io/learn\_gnuawk/awk-introduction.html?utm\_source=hackernewsletter\&utm\_medium=email\&utm\_term=fav}{
        \includegraphics[width=0.99\linewidth]{6.png}
    }
\paragraph{awk introduction
This chapter will give an overview of
awk syntax and some examples to show what kind of problems you could solve using
awk.}
 These features will be covered in depth in later, but you shouldn't skip this chapter.
Filtering
awk provides filtering capabilities like those supported by the
grep and
sed commands. As a programming language, there are additional nifty features as well. Similar to many command line utilities,
awk can accept input from both stdin and files.
\# sample stdin data \$ printf 'gatenapplenwhatnkiten' gate apple what kite \# same as: grep 'at' and sed -n '/at/p' \# filter lines containing 'at' \$ printf 'gatenapplenwhatnkiten' | awk '/at/' gate what \# same as: grep -v 'e' and sed -n '/e/!p' \# filter lines NOT containing 'e' \$ printf 'gatenapplenwhatnkiten' | awk '!/e/' what
By default,
awk automatically loops over the input content line by line. You can then use programming instructions to process those lines. As
awk is often used from the command line, many shortcuts are available to reduce the amount of typing needed.
In the above examples, a
\dots
\end{multicols}
\end{minipage}
\par\medskip
%\noindent\textcolor{red}{\rule{\linewidth}{0.2mm}}
\subsection{CT scans of coffee-making equipment}
\noindent\begin{minipage}[t]{0.19\linewidth}
\vspace{0pt}
\noindent\scshape\footnotesize
\\ {\scriptsize\faCalendar}\space 
2023-08-29
\\ {\scriptsize\faGlobe}\space 
scanofthemonth.com
\\ {\scriptsize\faThumbsOUp}\space 
\href{http://news.ycombinator.com/item?id=37341799\&utm\_term=comment}{370} 
\\ {\scriptsize\faComments}\space 
\href{http://news.ycombinator.com/item?id=37341799\&utm\_term=comment}{28} 
\end{minipage} 
\begin{minipage}[t]{0.80\linewidth}
\vspace{0pt}
\begin{multicols}{2}
    \href{https://www.scanofthemonth.com/scans/coffee?utm\_source=hackernewsletter\&utm\_medium=email\&utm\_term=fav}{
        \includegraphics[width=0.99\linewidth]{7.png}
    }
\paragraph{The history of coffee provides a rich index of global economic and cultural exchange going back thousands of years.}
 In the twentieth and twenty-first centuries, we’ve seen distinct “waves” of change in coffee consumption. Not only have roasting and brewing methods grown more refined, tools and accessories have also become increasingly precise in their engineering.
The First Wave marks the emergence of pre-ground, mass-produced coffee (think Folgers), while Second Wave introduces specialty coffee chains and espresso culture (like Starbucks). Third Wave delves into coffee as an artisanal craft, emphasizing bean origin and brewing methods (Blue Bottle, etc.). The emergent Fourth Wave pushes boundaries with scientific precision and sustainable practices, refining coffee’s production and enjoyment.
Our Neptune industrial CT scanner is the perfect companion for tracing this evolution, revealing how and why coffee tools and techniques stand the test of time.
Until the 20th century, coffee was primarily enjoyed in coffee shops. That all changed when Alfonso Bialetti invented the Moka Express
\dots
\end{multicols}
\end{minipage}
\par\medskip
%\noindent\textcolor{red}{\rule{\linewidth}{0.2mm}}
\subsection{Thoughts about what worked in math circles}
\noindent\begin{minipage}[t]{0.19\linewidth}
\vspace{0pt}
\noindent\scshape\footnotesize
\\ {\scriptsize\faUser}\space 
Halfspace
\\ {\scriptsize\faCalendar}\space 
2023-07-17
\\ {\scriptsize\faGlobe}\space 
buttondown.email
\\ {\scriptsize\faThumbsOUp}\space 
\href{http://news.ycombinator.com/item?id=37276502\&utm\_term=comment}{299} 
\\ {\scriptsize\faComments}\space 
\href{http://news.ycombinator.com/item?id=37276502\&utm\_term=comment}{14} 
\end{minipage} 
\begin{minipage}[t]{0.80\linewidth}
\vspace{0pt}
\begin{multicols}{2}
    \href{https://buttondown.email/j2kun/archive/thoughts-about-what-worked-in-math-circles/?utm\_source=hackernewsletter\&utm\_medium=email\&utm\_term=fav}{
        \includegraphics[width=0.99\linewidth]{8.png}
    }
\paragraph{After about 7 months of math circles with a group of 7- turning 8-year-old boys and girls, I decided to take a break to breathe and reflect on what worked and what didn't.}

It's interesting how big a gulf there is between what math topic you think will be interesting to a 7-year-old and what actually captures their attention. Let me start by giving some examples of things I thought would catch their interest but flopped.
Things I didn't think they'd like but they loved:
And then there were the problems I thought they would love, and they did.
Back in 2019 I wrote an article, Attention spans for math and stories in which I described how I have used storytelling with kids of various ages (not in a math context) to get them participating in activities and feeling welcome in a group. I tied it back to math,
To have good mathematical content revolving around stories, mathematicians should learn to tell stories well.
Somehow, though, my story-telling game was off during some of my math circles. I think through all my reading and learning about math circles, I had internalized a different vi
\dots
\end{multicols}
\end{minipage}
\par\medskip
%\noindent\textcolor{red}{\rule{\linewidth}{0.2mm}}
\subsection{Lie still in bed}
\noindent\begin{minipage}[t]{0.19\linewidth}
\vspace{0pt}
\noindent\scshape\footnotesize
\\ {\scriptsize\faUser}\space 
Ognjen Regoje
\\ {\scriptsize\faCalendar}\space 
2023-08-27
\\ {\scriptsize\faGlobe}\space 
ognjen.io
\\ {\scriptsize\faThumbsOUp}\space 
\href{http://news.ycombinator.com/item?id=37281060\&utm\_term=comment}{271} 
\\ {\scriptsize\faComments}\space 
\href{http://news.ycombinator.com/item?id=37281060\&utm\_term=comment}{50} 
\end{minipage} 
\begin{minipage}[t]{0.80\linewidth}
\vspace{0pt}
\begin{multicols}{2}
    \href{https://ognjen.io/lie-still-in-bed/?utm\_source=hackernewsletter\&utm\_medium=email\&utm\_term=fav}{
        \includegraphics[width=0.99\linewidth]{9.png}
    }
\paragraph{I found it very difficult to switch to a regular sleep, and wake, schedule after university.
I even started using one of the loudest and most annoying alarm clocks I could find.}
 (That sound still gives my university housemate flashbacks.)
In my search for ways to fix my sleep schedule, I came across a seemingly simple piece of advice:
Lie still in bed.
If I remember correctly, that article, like many others, suggested sleeping at the target time every day. To do that, it said, you should lie still in bed with your eyes closed.
It explained that most people fail not because they go to bed late, but because they play on their phones, watch TV or read a book. So, they go to bed earlier but still go to sleep late.
The logic made sense and I tried it. Of course, it didn’t work the first night. But it did in a couple of weeks.
Eventually, I was able to take that to an extreme and became a morning person.
Over the following \~10 years, I’ve successfully applied a generalized version of this advice to several other things.
And over that time I’ve realized three things about practice.
1. You c
\dots
\end{multicols}
\end{minipage}
\par\medskip
%\noindent\textcolor{red}{\rule{\linewidth}{0.2mm}}
\subsection{Anti-hype LLM Reading List}
\noindent\begin{minipage}[t]{0.19\linewidth}
\vspace{0pt}
\noindent\scshape\footnotesize
\\ {\scriptsize\faUser}\space 
Veekaybee
\\ {\scriptsize\faCalendar}\space 
2023-08-17
\\ {\scriptsize\faGithub}\space 
github.com
\\ {\scriptsize\faThumbsOUp}\space 
\href{http://news.ycombinator.com/item?id=37281020\&utm\_term=comment}{206} 
\\ {\scriptsize\faComments}\space 
\href{http://news.ycombinator.com/item?id=37281020\&utm\_term=comment}{10} 
\end{minipage} 
\begin{minipage}[t]{0.80\linewidth}
\vspace{0pt}
\begin{multicols}{2}
    \href{https://gist.github.com/veekaybee/be375ab33085102f9027853128dc5f0e?utm\_source=hackernewsletter\&utm\_medium=email\&utm\_term=fav}{
        \includegraphics[width=0.99\linewidth]{10.png}
    }
\paragraph{Anti-hype LLM reading list
Goals: Add links that are reasonable and good explanations of how stuff works. No hype and no vendor content if possible.}
 Practical first-hand accounts and experience preferred (super rare at this point).
My own notes from a few months back.
Background
- Survey of LLMS
- Self-attention and transformer networks
- What are embeddings
- The Illustrated Word2vec - A Gentle Intro to Word Embeddings in Machine Learning (YouTube)
- Catching up on the weird world of LLMS
Foundational Papers
- Attention is all you Need
- Scaling Laws for Neural Language Models
- BERT
- Language Models are Unsupervised Multi-Task Learners
- Training Language Models to Follow Instructions
- Language Models are Few-Shot Learners
Training Your Own
- Why host your own LLM?
- How to train your own LLMs
- Training Compute-Optimal Large Language Models
- Opt-175B Logbook
Algos
- The case for GZIP Classifiers and more on nearest neighbors algos
- Meta Recsys Using and extending Word2Vec
- The State of GPT (YouTube)
- What is ChatGPT doing and why does it work
- How is LlamaCPP Possible?
- On
\dots
\end{multicols}
\end{minipage}
\par\medskip
%\noindent\textcolor{red}{\rule{\linewidth}{0.2mm}}
\subsection{A video of the transformation of a single cell into a salamander}
\noindent\begin{minipage}[t]{0.19\linewidth}
\vspace{0pt}
\noindent\scshape\footnotesize
\\ {\scriptsize\faThumbsOUp}\space 
\href{http://news.ycombinator.com/item?id=37275401\&utm\_term=comment}{204} 
\\ {\scriptsize\faComments}\space 
\href{http://news.ycombinator.com/item?id=37275401\&utm\_term=comment}{19} 
\end{minipage} 
\begin{minipage}[t]{0.80\linewidth}
\vspace{0pt}
\begin{multicols}{2}
    \href{https://pmdvod.nationalgeographic.com/NG\_Video/772/995/1442844739770\_1550184269599\_1442865731523\_mp4\_video\_1024x576\_1632000\_primary\_audio\_eng\_3.mp4?utm\_source=hackernewsletter\&utm\_medium=email\&utm\_term=fav}{
        \includegraphics[width=0.99\linewidth]{11.png}
    }

\dots
\end{multicols}
\end{minipage}
\par\medskip
%\noindent\textcolor{red}{\rule{\linewidth}{0.2mm}}
\subsection{Why Htmx Does Not Have a Build Step}
\noindent\begin{minipage}[t]{0.19\linewidth}
\vspace{0pt}
\noindent\scshape\footnotesize
\\ {\scriptsize\faCalendar}\space 
2023-08-19
\\ {\scriptsize\faGlobe}\space 
htmx.org
\\ {\scriptsize\faThumbsOUp}\space 
\href{http://news.ycombinator.com/item?id=37265097\&utm\_term=comment}{175} 
\\ {\scriptsize\faComments}\space 
\href{http://news.ycombinator.com/item?id=37265097\&utm\_term=comment}{18} 
\end{minipage} 
\begin{minipage}[t]{0.80\linewidth}
\vspace{0pt}
\begin{multicols}{2}
    \href{https://htmx.org/essays/no-build-step/?utm\_source=hackernewsletter\&utm\_medium=email\&utm\_term=fav}{
        \includegraphics[width=0.99\linewidth]{12.png}
    }
\paragraph{A recurring question from some htmx contributors is why htmx isn’t written in TypeScript, or, for that matter, why htmx lacks any build step at all.}
 The full htmx source is a single 3,500-line JavaScript file; if you want to contribute to htmx, you do so by modifying the
htmx.js file, the same file that gets sent to browsers in production, give or take minification and compression.
I do not speak for the htmx project, but I have made a few nontrivial contributions to it, and have been a vocal advocate for retaining this no-build setup every time the issue has arisen. From my perspective, here’s why htmx does not have a build step.
The best reason to write a library in plain JavaScript is that it lasts forever. This is arguably JavaScript’s single most underrated feature. While I’m sure there are some corner cases, JavaScript from 1999 that ran in Netscape Navigator will run unaltered, alongside modern code, in Google Chrome downloaded yesterday. That is true for very few programming environments. It’s certainly not true for Python, or Java, or C, which all have versioning mechanisms 
\dots
\end{multicols}
\end{minipage}
\par\medskip
%\noindent\textcolor{red}{\rule{\linewidth}{0.2mm}}
\newpage
\section{\#Ask HN}

\begin{multicols}{2}
\raggedcolumns
\noindent\begin{minipage}{\linewidth}
\medskip
\subsection{Why did Python win?}
\textsc{\footnotesize
{\scriptsize\faCalendar}\space 
2023-08-29 
{\scriptsize\faThumbsOUp}\space 
\href{}{565} 
{\scriptsize\faComments}\space 
\href{}{243} 
}
\par\medskip\noindent
\href{https://news.ycombinator.com/item?id=37308747\&utm\_source=hackernewsletter\&utm\_medium=email\&utm\_term=ask\_hn}{
    \includegraphics[width=0.99\linewidth]{13.png}
}
\end{minipage}
\paragraph{}
\textbf{I started programming in \~2013 in JavaScript. I’ve since learned and tried a handful of languages, including Python, but JavaScript was always my favorite.}
\paragraph{}
 Just within the last year I learned Ruby, and I was blown away by how fun and easy to use it is. At the present time, I’m starting all my new projects in Ruby.
My impression is that in the ‘00s, Python and Ruby were both relatively new, dynamically typed, “English-like” languages. And for a while these languages had similar popularity.
Now Ruby is still very much alive; there are plenty of Rails jobs available and exciting things happening with Ruby itself. But Python has become a titan in the last ten years. It has continued to grow exponentially and Ruby has not.
I can guess as to why (Python’s math libraries, numpy and pandas make it appealing to academics; Python is simpler and possibly easier to learn; Rails was so popular that it was synonymous with Ruby) but I wasn’t paying attention at that time. So I’m interested in hearing from some of the older programmers about why Ruby has stalled out and Python has become possibly th
\dots\par
%\par\noindent\textcolor{red}{\rule{\linewidth}{0.2mm}}
\noindent\begin{minipage}{\linewidth}
\medskip
\subsection{Where do I find good code to read?}
\textsc{\footnotesize
{\scriptsize\faCalendar}\space 
2023-08-24 
{\scriptsize\faThumbsOUp}\space 
\href{}{166} 
{\scriptsize\faComments}\space 
\href{}{66} 
}
\par\medskip\noindent
\href{https://news.ycombinator.com/item?id=37248002\&utm\_source=hackernewsletter\&utm\_medium=email\&utm\_term=ask\_hn}{
    \includegraphics[width=0.99\linewidth]{14.png}
}
\end{minipage}
\paragraph{}
\textbf{I feel like I get a lot more out of messing with / hacking on code than I do from reading it.}
\paragraph{}
 I'm sure people vary, but I've got loads more out of open source contributions to sometimes small projects, and not very much out of trying to do something like read the code for the Glasgow Haskel Compiler or something.
So for code to read, I think having an in is crucial, at least for me. So I'd say, find a cool, maybe small open source project, look at the issue tracker if it has one, and try to implement something. You'll only really know if it's good code (or why) after you start trying to change it.
reply
I like this and it is true. In addition, I have found that good code also tends to survive refractors despite having the quality of being easy to change.
The Ask HN post specifically mentioned good code but what follows are some (of my subjective) thoughts about the benefits of reading bad code. Bad code is haphazard and varied and good code is “samey”. You are a pattern matcher and this is part of your training. Good code will make more of an impact on your understanding if you have 
\dots\par
%\par\noindent\textcolor{red}{\rule{\linewidth}{0.2mm}}
\noindent\begin{minipage}{\linewidth}
\medskip
\subsection{What's your favourite hobby and how did it start?}
\textsc{\footnotesize
{\scriptsize\faCalendar}\space 
2023-08-31 
{\scriptsize\faThumbsOUp}\space 
\href{}{49} 
{\scriptsize\faComments}\space 
\href{}{56} 
}
\par\medskip\noindent
\href{https://news.ycombinator.com/item?id=37342013\&utm\_source=hackernewsletter\&utm\_medium=email\&utm\_term=ask\_hn}{
    \includegraphics[width=0.99\linewidth]{15.png}
}
\end{minipage}
\paragraph{}
\textbf{https://www.ableton.com/en/live/
https://www.youtube.com/watch?v=0iuRsiKtObw
https://www.youtube.com/watch?v=mxY0x1i3XhY
https://youtube.com/playlist?list=PL9oiyAGA6zOTSPR5-ttojODT4...}
\paragraph{}

Getting some teenage engineering pocket operators is also something I'd strongly recommend. They're affordable and fun!
Check out this 8 year old building a tune with them: https://www.youtube.com/watch?v=IhFIUdICYSA
reply
Btw thanks for the links, they look super helpful.
I also learned ableton without any resources, but back in 2003 when it was a simpler product. It begs to be tinkered with (I think that’s the draw of it) but it’s a deep bit of software.
I started after enrolling my children in it, and then figuring "why the hell not, I should train too."
I only got out of it because of a combination of A. lack of time, and B. injuries (note: the injuries were more from mountain biking than from BJJ, but I did tear a hip adductor muscle once in BJJ training).
Get a good pair of binoculars, a bird sound recognition app for your mobile[1], and a bird field guide (as a European, I prefer [2]), and you'r
\dots\par
%\par\noindent\textcolor{red}{\rule{\linewidth}{0.2mm}}
\end{multicols}

\newpage
\section{\#Classifieds}

\begin{multicols}{2}
\raggedcolumns
\noindent\begin{minipage}{\linewidth}
\medskip
\subsection{DemoHop: Better large team meetings when in-person isn't an option}
\textsc{\footnotesize
{\scriptsize\faCalendar}\space 
2023-07-14 
{\scriptsize\faGlobe}\space 
demohop.com 
{\scriptsize\faComments}\space 
\href{}{None} 
}
\par\medskip\noindent
\href{https://demohop.com}{
    \includegraphics[width=0.99\linewidth]{16.png}
}
\end{minipage}
\paragraph{}
The ultimate solution for connecting your distributed team
Restore connections and boost collaboration with online demo days, showcases and other events that break through the rigidity of typical video-based meetings.
DemoHop overcomes these distributed work challenges
Rejuvenate your team's connections with DemoHop
Before a DemoHop: Relationships, teams and ideas are isolated which leads to lackluster collaboration and low innovation.
After a DemoHop: Employees and teams rebuild their informal networks as ideas spread and new collaborations emerge.
Our social physics make the difference
Unlike general purpose video calls, the social physics built into DemoHop lead to genuine connection and information sharing.
It’s all about how and why people interact with each other. DemoHop embeds these social physics into the heart of the experience. This way people engage with each other regardless of their location, seniority, role, tenure or personality preferences.
Easy to run
1. Setup
Anyone can create a polished DemoHop event with ease. The event admin can delegate each booth's setup to a 
\dots\par
%\par\noindent\textcolor{red}{\rule{\linewidth}{0.2mm}}
\noindent\begin{minipage}{\linewidth}
\medskip
\subsection{Redact PII for OpenAI with a simple proxy}
\textsc{\footnotesize
{\scriptsize\faUser}\space 
Maitham Dib Founder; Vital 
{\scriptsize\faCalendar}\space 
2023-07-09 
{\scriptsize\faComments}\space 
\href{}{None} 
}
\par\medskip\noindent
\href{https://evervault.com/use-cases/ai-privacy?utm\_source=newsletter\&utm\_medium=email\&utm\_campaign=hackernewsletter}{
    \includegraphics[width=0.99\linewidth]{17.png}
}
\end{minipage}
\paragraph{}
\textbf{AI Privacy
Securely integrate with OpenAI without sharing sensitive customer PII or compromising on AI output.}
\paragraph{}

Privacy proxy forthird-party LLMs
Only share necessary data with LLMs
Leverage the power of a third-party LLM in your product, without compromising on data security. Using Evervault’s Relay Redaction ensures your customers’ PII is removed before reaching a service like OpenAI.
Build customer trust compliantly
Easily assure your customers that their sensitive data isn’t being shared with black-box third-party LLM providers, like OpenAI. Protect their data, and maintain your existing compliance status in HIPAA, GDPR, CCPA and more.
Minimal infrastructure changes
No need to re-architect your integration or data flow. Simply put Evervault Relay in front of the request and the PII is redacted before sharing.
Powerful redaction for any data workflow
Detect, redact, and replace PII in unstructured data – like support tickets, chat transcripts, and patient medical records – before sharing with a third-party LLM.
Proxy your existing network requests to any LLM API via Evervault Relay
\dots\par
%\par\noindent\textcolor{red}{\rule{\linewidth}{0.2mm}}
\noindent\begin{minipage}{\linewidth}
\medskip
\subsection{Machines Ethics Podcast - conversations on AI, ML and Ethics}
\textsc{\footnotesize
{\scriptsize\faComments}\space 
\href{}{None} 
}
\par\medskip\noindent
\href{https://www.machine-ethics.net}{
    \includegraphics[width=0.99\linewidth]{18.png}
}
\end{minipage}
\paragraph{}
\textbf{Machine Ethics Podcast
Conversations on AI Ethics:
Interrogating technology, artificial intelligence and society.}
\paragraph{}

This podcast brings together interviews with academics, authors, business leaders, designers and engineers on the subject of autonomous algorithms, artificial intelligence, machine learning, and technology's impact on society. Join in the conversation with us by getting in touch via email here or following us on Twitter and Instagram.
Listen to episodes here or subscribe on your favourite podcast app - e.g. iTunes, Spotify, Google podcasts, Stitcher and TuneIn. You can also support us by donating to the show on Patreon or ask a questions here.
The machine ethics podcast is made by www.ethicalby.design
79. Taming Uncertainty with Roger Spitz
This time we chat with Roger Spitz about how to think about the future, what does a futurist do? Thriving with disruption, a chief existential officer, virtuous inflection points, delegating too much authority / decision making, our inappropriate education system
78. Design and AI with Nadia Piet
This episode Nadia and I chat about how
\dots\par
%\par\noindent\textcolor{red}{\rule{\linewidth}{0.2mm}}
\noindent\begin{minipage}{\linewidth}
\medskip
\subsection{Analytics without bothering your data \& analytics engineers}
\textsc{\footnotesize
{\scriptsize\faCalendar}\space 
2013-07-01 
{\scriptsize\faComments}\space 
\href{}{None} 
}
\par\medskip\noindent
\href{https://welcome.humanix.app/beta-launch/}{
    \includegraphics[width=0.99\linewidth]{19.png}
}
\end{minipage}
\paragraph{}
\textbf{Breeze through setup. No more waiting on the Data Team to get around to your asks.
No SQL. No Python. If you know your way around a spreadsheet then you're already expert level using our platform.}
\paragraph{}

We're your analytics experts now and available for calls, Zooms, emails, date nights - whatever you need.
So we've made something great...but we need to test it out in the wild with real data. So expect hiccups, longer onboarding, some embarrassing moments but in exchange you'll get our undivided attention, affection, and the lowest price you'll ever see from us.
Available for a limited time!
\$100/100K rows synced per month
Not available until general release.
\$200/100K rows synced per month
Not available until general release.
\$150/100K rows synced per month
© Copyright • Humanix
\dots\par
%\par\noindent\textcolor{red}{\rule{\linewidth}{0.2mm}}
\noindent\begin{minipage}{\linewidth}
\medskip
\subsection{Buy a classified ad}
\textsc{\footnotesize
{\scriptsize\faCalendar}\space 
2023-04-01 
{\scriptsize\faComments}\space 
\href{}{None} 
}
\par\medskip\noindent
\href{https://airtable.com/shr74MwRpJSHRWsFl}{
    \includegraphics[width=0.99\linewidth]{20.png}
}
\end{minipage}
\paragraph{}
Hacker Newsletter Classifieds Alert Lorem ipsum Okay
\dots\par
%\par\noindent\textcolor{red}{\rule{\linewidth}{0.2mm}}
\end{multicols}

\newpage
\section{\#Show HN}

\begin{multicols}{2}
\raggedcolumns
\noindent\begin{minipage}{\linewidth}
\medskip
\subsection{I automated half of my typing}
\textsc{\footnotesize
{\scriptsize\faUser}\space 
Eschluntz 
{\scriptsize\faCalendar}\space 
2023-08-31 
{\scriptsize\faGithub}\space 
github.com 
{\scriptsize\faThumbsOUp}\space 
\href{http://news.ycombinator.com/item?id=37326870\&utm\_term=comment}{763} 
{\scriptsize\faComments}\space 
\href{http://news.ycombinator.com/item?id=37326870\&utm\_term=comment}{88} 
}
\par\medskip\noindent
\href{https://github.com/eschluntz/compress?utm\_source=hackernewsletter\&utm\_medium=email\&utm\_term=show\_hn}{
    \includegraphics[width=0.99\linewidth]{21.png}
}
\end{minipage}
\paragraph{}
\textbf{Compress
This is a tool for automatically creating typing shortcuts from a corpus of your own writing!}
\paragraph{}
 I use these shortcuts mainly for email and slack:
This repo parses a corpus of text and suggest what shortcuts you should use to save the most letters while typing. It then generates config files for Autokey, a linux program that implements keyboard shortcuts!
It also contains a tool for optionally parsing a Slack Data Export of your messages to create a corpus.
What phrases should I abbreviate?
The code looks through the corpus to find common n-grams that can be replaced with much shorter phrases. The suggestions are ranked by
[characters saved] * [frequency of phrase].
I was surprised that very short and frequent words topped this list, such as
the -> t, instead of longer phrases that I use a lot, such as
what do you think -> wdytk.
Just reading through the results was amusing to see how repetitive some of my writing is :)
How to pick abbreviations?
This is largely preferences and heuristics to try to generate memorable abbreviations for different phrases. Some of my design philos
\dots\par
%\par\noindent\textcolor{red}{\rule{\linewidth}{0.2mm}}
\noindent\begin{minipage}{\linewidth}
\medskip
\subsection{Open-source obsidian.md sync server}
\textsc{\footnotesize
{\scriptsize\faCalendar}\space 
2023-08-24 
{\scriptsize\faThumbsOUp}\space 
\href{}{245} 
{\scriptsize\faComments}\space 
\href{}{17} 
}
\par\medskip\noindent
\href{https://news.ycombinator.com/item?id=37247767\&utm\_source=hackernewsletter\&utm\_medium=email\&utm\_term=show\_hn}{
    \includegraphics[width=0.99\linewidth]{22.png}
}
\end{minipage}
\paragraph{}
\textbf{https://github.com/acheong08/obsidian-sync
Hello HN,
I'm a recent high school graduate and can't afford \$8 per month for the official sync service, so I tried my hand at replicating the server.}
\paragraph{}

It's still missing a few features, such as file recovery and history, but the basic sync is working.
To the creators of Obsidian.md: I'm probably violating the TOS, and I'm sorry. I'll take down the repository if asked. It's not ready for production and is highly inefficient; Not competition, so I hope you'll be lenient.
Impressive! It's fun to see the diversity of ways people sync/backup their Obsidian files. The nice thing about storing all your notes on your device is that it makes it possible to move and edit your Markdown files in many different ways. That diversity of solutions is what makes the ecosystem of Markdown tools resilient over the long term.
There are already a handful of tools that allow you to sync your notes for free, including Git, Syncthing, and some other options more specialized for Obsidian (see community plugins).
Obsidian is a small company, we're not VC backed (100
\dots\par
%\par\noindent\textcolor{red}{\rule{\linewidth}{0.2mm}}
\noindent\begin{minipage}{\linewidth}
\medskip
\subsection{Fairphone 5}
\textsc{\footnotesize
{\scriptsize\faCalendar}\space 
2023-02-27 
{\scriptsize\faGlobe}\space 
fairphone.com 
{\scriptsize\faThumbsOUp}\space 
\href{http://news.ycombinator.com/item?id=37319626\&utm\_term=comment}{243} 
{\scriptsize\faComments}\space 
\href{http://news.ycombinator.com/item?id=37319626\&utm\_term=comment}{34} 
}
\par\medskip\noindent
\href{https://shop.fairphone.com/fairphone-5?utm\_source=hackernewsletter\&utm\_medium=email\&utm\_term=show\_hn}{
    \includegraphics[width=0.99\linewidth]{23.png}
}
\end{minipage}
\paragraph{}
\textbf{FAIRPHONE 5
Designed for you. Made fair.}
\paragraph{}

Everything Fairphone 5 in under a minute
Our most sustainable
phone yet
Premium European Design in 3 colors
Matte black, sky blue or the transparent edition - with a back cover made from 100 recycled plastic.
Stunning pictures \& videos
Capture every moment in beautiful, dynamic detail with 3 premium 50 megapixel cameras
5 YEAR WARRANTY
Your Fairphone 5 is covered, so you don’t have to worry. It’s that simple. Remember to register your warranty here.
SOFTWARE UPDATES UNTIL 2031
That’s 8 years of support and security guaranteed. You can count on us to keep your phone running smoothly.
More than 70 fair or recycled focus materials
Gold, cobalt, lithium, plastics and many more - That's a groundbreaking innovation (literally).
EASY TO REPAIR
Nothing in the Fairphone 5 is glued shut. You can just fix it yourself and make it last. It’s that easy.
A dynamic OLED screen
Stunning picture quality with super smooth motion performance - bright and vibrant even in direct sunlight.
Designed for you.
Futureproof performance until 2031
Your Fairphone 5 is fa
\dots\par
%\par\noindent\textcolor{red}{\rule{\linewidth}{0.2mm}}
\noindent\begin{minipage}{\linewidth}
\medskip
\subsection{n8n.io - A powerful workflow automation tool}
\textsc{\footnotesize
{\scriptsize\faCalendar}\space 
2022-01-01 
{\scriptsize\faGlobe}\space 
n8n.io 
{\scriptsize\faThumbsOUp}\space 
\href{http://news.ycombinator.com/item?id=37274052\&utm\_term=comment}{234} 
{\scriptsize\faComments}\space 
\href{http://news.ycombinator.com/item?id=37274052\&utm\_term=comment}{20} 
}
\par\medskip\noindent
\href{https://n8n.io?utm\_source=hackernewsletter\&utm\_medium=email\&utm\_term=show\_hn}{
    \includegraphics[width=0.99\linewidth]{24.png}
}
\end{minipage}
\paragraph{}
\textbf{Full source code available:
Audit, tweak, and fork our codebase to suit your needs.
Workflow automation for technical people
Your days spent slogging through a spaghetti of scripts are over.}
\paragraph{}

Use JavaScript when you need flexibility and UI for everything else.
Connect APIs with no code to automate basic tasks. Or write vanilla Javascript when you need to manipulate complex data.
You can implement multiple triggers. Branch and merge your workflows. And even pause flows to wait for external events.
Interface easily with any API or service with custom HTTP requests.
Avoid breaking live workflows by separating dev and prod environments with unique sets of auth data.
n8n nodes let you process data at scale with a built-in iteration functionality in every node.
See the execution data of every workflow. Integrate error nodes into workflows to catch errors. And rerun individual nodes to test fixes fast.
Fully on-prem for those that need the security
Save time building customer integrations. Engineer faster POCs. And keep your customer-specific functionality separate from product. All without 
\dots\par
%\par\noindent\textcolor{red}{\rule{\linewidth}{0.2mm}}
\noindent\begin{minipage}{\linewidth}
\medskip
\subsection{Youtube2Webpage: Create Websites with Text from Videos}
\textsc{\footnotesize
{\scriptsize\faUser}\space 
Obra 
{\scriptsize\faCalendar}\space 
2023-09-01 
{\scriptsize\faGithub}\space 
github.com 
{\scriptsize\faThumbsOUp}\space 
\href{http://news.ycombinator.com/item?id=37333195\&utm\_term=comment}{191} 
{\scriptsize\faComments}\space 
\href{http://news.ycombinator.com/item?id=37333195\&utm\_term=comment}{17} 
}
\par\medskip\noindent
\href{https://github.com/obra/Youtube2Webpage?utm\_source=hackernewsletter\&utm\_medium=email\&utm\_term=show\_hn}{
    \includegraphics[width=0.99\linewidth]{25.png}
}
\end{minipage}
\paragraph{}
\textbf{Youtube-to-Webpage
Youtube-to-Webpage is a Perl script to create a webpage from a Youtube video with a transcript generated from the video's closed captions paired with screenshots of the video.
.}
\paragraph{}
/yt-to-webpage.pl project-name "videoURL"
Dependencies
The project is built upon:
Using
To use, run the Perl script with a name for the folder to create, and the video URL. For example:
./yt-to-webpage.pl project-name "https://www.youtube.com/watch?v=jNQXAC9IVRw"
Output
Running the script create a repository according to the following structure:
project-name ├── images │ └── (…).jpg ├── video.vtt ├── video.webm ├── index.html └── styles.css
- The index.html file is the generated webpage.
- The images directory contains all the screenshots, named according to their timeframe
hours-minutes-seconds-milliseconds.jpg.
- The vtt file contains the captions.
- The webm file contains the video.
- The css file styles the webpage.
Example
You can see an example at https://obra.github.io/Youtube2Webpage/example/
\dots\par
%\par\noindent\textcolor{red}{\rule{\linewidth}{0.2mm}}
\noindent\begin{minipage}{\linewidth}
\medskip
\subsection{Shimmer – ADHD coaching for adults, now on web}
\textsc{\footnotesize
{\scriptsize\faUser}\space 
Chris Wang; Shimmer Co-founder; CEO August 
{\scriptsize\faCalendar}\space 
2023-08-10 
{\scriptsize\faThumbsOUp}\space 
\href{http://news.ycombinator.com/item?id=37252231\&utm\_term=comment}{154} 
{\scriptsize\faComments}\space 
\href{http://news.ycombinator.com/item?id=37252231\&utm\_term=comment}{33} 
}
\par\medskip\noindent
\href{https://www.shimmer.care/blog/web-launch-adhd-coaching?utm\_source=hackernewsletter\&utm\_medium=email\&utm\_term=show\_hn}{
    \includegraphics[width=0.99\linewidth]{26.png}
}
\end{minipage}
\paragraph{}
\textbf{Shimmer was born from my personal frustration when I was diagnosed last year \& wasn’t able to find an affordable, quality ADHD coach.}
\paragraph{}
 True hyperfocus mode, I joined up with the top ADHD experts, the ADHD community, and an incredible team to build the first Affordable ADHD Coaching Platform, built by and for people with ADHD.
This last year has been a whirlwind of an adventure to say the least. Since then, we’ve built an incredible ADHD community of 60K+, onboarded 20+ coaches, launched our ADHD Coaching Programme, and served over 1,000 members through it.
A bit on how Shimmer works:
Contrary to the prevailing notion of pushing mobile usage and competing for attention, our core purpose has always centered around addressing the unique needs of people with ADHD and prioritizing coaching outcomes. So we ask ourselves the question: What are the key obstacles or voids preventing our members from fully realizing their coaching goals? We then reverse-engineer solutions from these challenges. Here are a few primary hurdles we tackled and how the web platform elegantly resolves them:
For those
\dots\par
%\par\noindent\textcolor{red}{\rule{\linewidth}{0.2mm}}
\end{multicols}

\newpage
\section{\#Code}

\begin{multicols}{2}
\raggedcolumns
\noindent\begin{minipage}{\linewidth}
\medskip
\subsection{Code Llama, a state-of-the-art large language model for coding}
\textsc{\footnotesize
{\scriptsize\faThumbsOUp}\space 
\href{http://news.ycombinator.com/item?id=37248844\&utm\_term=comment}{644} 
{\scriptsize\faComments}\space 
\href{http://news.ycombinator.com/item?id=37248844\&utm\_term=comment}{2} 
}
\par\medskip\noindent
\href{https://ai.meta.com/blog/code-llama-large-language-model-coding/?utm\_source=hackernewsletter\&utm\_medium=email\&utm\_term=code}{
    \includegraphics[width=0.99\linewidth]{27.png}
}
\end{minipage}
\paragraph{}

\dots\par
%\par\noindent\textcolor{red}{\rule{\linewidth}{0.2mm}}
\noindent\begin{minipage}{\linewidth}
\medskip
\subsection{Slack’s migration to a cellular architecture}
\textsc{\footnotesize
{\scriptsize\faUser}\space 
Stephan Zuercher 
{\scriptsize\faCalendar}\space 
2023-08-22 
{\scriptsize\faGlobe}\space 
slack.engineering 
{\scriptsize\faThumbsOUp}\space 
\href{http://news.ycombinator.com/item?id=37274871\&utm\_term=comment}{394} 
{\scriptsize\faComments}\space 
\href{http://news.ycombinator.com/item?id=37274871\&utm\_term=comment}{27} 
}
\par\medskip\noindent
\href{https://slack.engineering/slacks-migration-to-a-cellular-architecture/?utm\_source=hackernewsletter\&utm\_medium=email\&utm\_term=code}{
    \includegraphics[width=0.99\linewidth]{28.png}
}
\end{minipage}
\paragraph{}
\textbf{Summary
In recent years, cellular architectures have become increasingly popular for large online services as a way to increase redundancy and limit the blast radius of site failures.}
\paragraph{}
 In pursuit of these goals, we have migrated the most critical user-facing services at Slack from a monolithic to a cell-based architecture over the last 1.5 years. In this series of blog posts, we’ll discuss our reasons for embarking on this massive migration, illustrate the design of our cellular topology along with the engineering trade-offs we made along the way, and talk about our strategies for successfully shipping deep changes across many connected services.
Background: the incident
At Slack, we conduct an incident review after each notable service outage. Below is an excerpt from our internal report summarizing one such incident and our findings:
At 11:45am PDT on 2021-06-30, our cloud provider experienced a network disruption in one of several availability zones in our U.S. East Coast region, where the majority of Slack is hosted. A network link that connects one availability zone with several 
\dots\par
%\par\noindent\textcolor{red}{\rule{\linewidth}{0.2mm}}
\noindent\begin{minipage}{\linewidth}
\medskip
\subsection{Fomos: Experimental OS, built with Rust}
\textsc{\footnotesize
{\scriptsize\faUser}\space 
Ruddle 
{\scriptsize\faCalendar}\space 
2023-08-30 
{\scriptsize\faGithub}\space 
github.com 
{\scriptsize\faThumbsOUp}\space 
\href{http://news.ycombinator.com/item?id=37316309\&utm\_term=comment}{380} 
{\scriptsize\faComments}\space 
\href{http://news.ycombinator.com/item?id=37316309\&utm\_term=comment}{35} 
}
\par\medskip\noindent
\href{https://github.com/Ruddle/Fomos?utm\_source=hackernewsletter\&utm\_medium=email\&utm\_term=code}{
    \includegraphics[width=0.99\linewidth]{29.png}
}
\end{minipage}
\paragraph{}
\textbf{Dear HN readers
I am actively looking for remote software engineering work. mail@thomassimon.dev You can support this night time project by hiring me for a day time job !}
\paragraph{}

Fomos
Experimental OS, built with Rust
output.mp4
Fun fact: there are 3 apps running in the video. A background app, a cursor app, and a console app.
Why
I wanted to experiment with Non-Unix OS ideas.
Exo-kernels are interesting, but it is mostly a theory. This project helps me understand the challenges involved in that pattern.
OS development is extremely hard, Rust makes it more bearable.
Features
- Has a graphical output
- Dynamic allocation
- Load and run concurrent apps
- All apps run in an async loop
- Support Virtio mouse and keyboard (drivers are async tasks)
- Cooperative scheduling (apps yield control as much as possible)
- No context switches once booted
- Nearly support Virgl ™
There is 5 examples of apps in this repo named
app\_*, some in Rust, one in C.
The kernel is in
bootloader.
What is unique
The signature of an app in Fomos:
pub extern "C" fn \_start(ctx: \&mut Context) -> i32
Apps do not need a stan
\dots\par
%\par\noindent\textcolor{red}{\rule{\linewidth}{0.2mm}}
\noindent\begin{minipage}{\linewidth}
\medskip
\subsection{Fortran}
\textsc{\footnotesize
{\scriptsize\faCalendar}\space 
2022-02-01 
{\scriptsize\faGlobe}\space 
fortran-lang.org 
{\scriptsize\faThumbsOUp}\space 
\href{http://news.ycombinator.com/item?id=37291504\&utm\_term=comment}{289} 
{\scriptsize\faComments}\space 
\href{http://news.ycombinator.com/item?id=37291504\&utm\_term=comment}{25} 
}
\par\medskip\noindent
\href{https://fortran-lang.org/en/index.html?utm\_source=hackernewsletter\&utm\_medium=email\&utm\_term=code}{
    \includegraphics[width=0.99\linewidth]{30.png}
}
\end{minipage}
\paragraph{}
The Fortran Programming Language\#
Fortran
High-performance parallel programming language
Features
High performance
Fortran has been designed from the ground up for computationally intensive applications in science and engineering. Mature and battle-tested compilers and libraries allow you to write code that runs close to the metal, fast.
Statically and strongly typed
Fortran is statically and strongly typed, which allows the compiler to catch many programming errors early on for you. This also allows the compiler to generate efficient binary code.
Easy to learn and use
Fortran is a relatively small language that is surprisingly easy to learn and use. Expressing most mathematical and arithmetic operations over large arrays is as simple as writing them as equations on a whiteboard.
Versatile
Fortran allows you to write code in a style that best fits your problem: imperative, procedural, array-oriented, object-oriented, or functional.
Natively parallel
Fortran is a natively parallel programming language with intuitive array-like syntax to communicate data between CPUs. You can run almos
\dots\par
%\par\noindent\textcolor{red}{\rule{\linewidth}{0.2mm}}
\noindent\begin{minipage}{\linewidth}
\medskip
\subsection{Fish – A friendly interactive shell}
\textsc{\footnotesize
{\scriptsize\faUser}\space 
Fish-Shell 
{\scriptsize\faCalendar}\space 
2023-09-01 
{\scriptsize\faGithub}\space 
github.com 
{\scriptsize\faThumbsOUp}\space 
\href{http://news.ycombinator.com/item?id=37272611\&utm\_term=comment}{248} 
{\scriptsize\faComments}\space 
\href{http://news.ycombinator.com/item?id=37272611\&utm\_term=comment}{32} 
}
\par\medskip\noindent
\href{https://github.com/fish-shell/fish-shell?utm\_source=hackernewsletter\&utm\_medium=email\&utm\_term=code}{
    \includegraphics[width=0.99\linewidth]{31.png}
}
\end{minipage}
\paragraph{}
\textbf{fish - the friendly interactive shell
fish is a smart and user-friendly command line shell for macOS, Linux, and the rest of the family.}
\paragraph{}
 fish includes features like syntax highlighting, autosuggest-as-you-type, and fancy tab completions that just work, with no configuration required.
For downloads, screenshots and more, go to https://fishshell.com/.
Quick Start
fish generally works like other shells, like bash or zsh. A few important differences can be found at https://fishshell.com/docs/current/tutorial.html by searching for the magic phrase “unlike other shells”.
Detailed user documentation is available by running
help within
fish, and also at https://fishshell.com/docs/current/index.html
Getting fish
macOS
fish can be installed:
- using Homebrew:
brew install fish
- using MacPorts:
sudo port install fish
- using the installer from fishshell.com
- as a standalone app from fishshell.com
Note: The minimum supported macOS version is 10.10 "Yosemite".
Packages for Linux
Packages for Debian, Fedora, openSUSE, and Red Hat Enterprise Linux/CentOS are available from the openSUSE Build Serv
\dots\par
%\par\noindent\textcolor{red}{\rule{\linewidth}{0.2mm}}
\noindent\begin{minipage}{\linewidth}
\medskip
\subsection{Email Authentication: A Developer's Guide}
\textsc{\footnotesize
{\scriptsize\faCalendar}\space 
2023-08-25 
{\scriptsize\faGlobe}\space 
resend.com 
{\scriptsize\faThumbsOUp}\space 
\href{http://news.ycombinator.com/item?id=37263708\&utm\_term=comment}{209} 
{\scriptsize\faComments}\space 
\href{http://news.ycombinator.com/item?id=37263708\&utm\_term=comment}{11} 
}
\par\medskip\noindent
\href{https://resend.com/blog/email-authentication-a-developers-guide?utm\_source=hackernewsletter\&utm\_medium=email\&utm\_term=code}{
    \includegraphics[width=0.99\linewidth]{32.png}
}
\end{minipage}
\paragraph{}
\textbf{Learn the importance of SPF, DKIM, DMARC, and BIMI in ensuring email delivery.}
\paragraph{}

Proper email authentication can be the difference between reaching the human or the spam folder, but it is often overlooked or misunderstood.
Think of your emails as a startup getting into a competitive accelerator program.
Competitive startup programs will receive 10's of thousands of applications. Their first step is to see which of these applications can be thrown out without being considered.
SPF (Sender Policy Framework) is similar. It's the first triage of the emails coming to an inbox, checking to make sure that each email should even be considered for delivery.
The DNS record for SPF declares a list of origins (servers) that are allowed to send email for this domain, and the inbox will confirm that the message they received matches one of them. If a server isn't on the list, it's like an application being tossed out because it wasn't fully filled or the business idea is illegal.
Every domain or subdomain can only have one SPF policy, and policies on the root/apex domain (domain.com) are not applied
\dots\par
%\par\noindent\textcolor{red}{\rule{\linewidth}{0.2mm}}
\end{multicols}

\newpage
\section{\#Data}

\begin{multicols}{2}
\raggedcolumns
\noindent\begin{minipage}{\linewidth}
\medskip
\subsection{Big Ass Data Broker Opt-Out List}
\textsc{\footnotesize
{\scriptsize\faUser}\space 
Yaelwrites 
{\scriptsize\faCalendar}\space 
2023-08-29 
{\scriptsize\faGithub}\space 
github.com 
{\scriptsize\faThumbsOUp}\space 
\href{http://news.ycombinator.com/item?id=37265346\&utm\_term=comment}{154} 
{\scriptsize\faComments}\space 
\href{http://news.ycombinator.com/item?id=37265346\&utm\_term=comment}{15} 
}
\par\medskip\noindent
\href{https://github.com/yaelwrites/Big-Ass-Data-Broker-Opt-Out-List?utm\_source=hackernewsletter\&utm\_medium=email\&utm\_term=data}{
    \includegraphics[width=0.99\linewidth]{33.png}
}
\end{minipage}
\paragraph{}
\textbf{Big Ass Data Broker Opt-Out List
|Symbols||Meanings|
|☠||high priority|
|🎫||requires driver’s license (cross out your ID \#!}
\paragraph{}
)|
|📫||must use snail mail|
|💰||site charges money for access or removal (whaaaat?)|
This list was started on September 29, 2017 and was most recently updated in May 2023 to add information on sites that require you to click links sent via email or to receive an automated call and enter a four-digit number on your phone in order to complete an opt-out request.
Please send corrections or updates to yael@yaelwrites.com, or file a pull request. Screenshots in emails are incredibly helpful. I will add opt-outs where users can verify that the data broker has their information before providing it, and where removal is not limited to GDPR/CCPA/etc.
Disclaimers: Some of these opt-outs take a long time to go through. Sometimes, information is pulled from other sources and you’ll need to opt out multiple times for the same site. Data brokers come and go (and are bought out by others), and they also often change their opt-out pages. I try to update this \~every six months, b
\dots\par
%\par\noindent\textcolor{red}{\rule{\linewidth}{0.2mm}}
\noindent\begin{minipage}{\linewidth}
\medskip
\subsection{File Attachments: Databases can now store files and images}
\textsc{\footnotesize
{\scriptsize\faCalendar}\space 
2023-08-30 
{\scriptsize\faThumbsOUp}\space 
\href{http://news.ycombinator.com/item?id=37324370\&utm\_term=comment}{128} 
{\scriptsize\faComments}\space 
\href{http://news.ycombinator.com/item?id=37324370\&utm\_term=comment}{17} 
}
\par\medskip\noindent
\href{https://xata.io/blog/file-attachments?utm\_source=hackernewsletter\&utm\_medium=email\&utm\_term=data}{
    \includegraphics[width=0.99\linewidth]{34.png}
}
\end{minipage}
\paragraph{}
\textbf{Extending a database record to include a new file column type with a global CDN, common security boundaries, and image transformations.}
\paragraph{}

Written by
Sorin Toma
Published on
August 30, 2023
Today, as part of our launch week, we’re beyond excited to announce a feature that we've wanted to add ever since we started Xata: File Attachments. Think of it as having a new database column type where you can store files of any size, and behind the scenes they are stored in AWS S3 and cached through a global CDN. Files simply become a part of a database record. For example, they respect the same security boundary -- if you can access a record, you can also access its attached files. Image file types also get some extra functionality allowing you to request them at any size and style with built-in transformations. With this release, we aim to simplify your application architecture and reduce the number of services you need to manage.
In this blog we'll dive into the capabilities released today and architecture behind our implementation. We hope you enjoy the read as much as we did building it 😃
Bot
\dots\par
%\par\noindent\textcolor{red}{\rule{\linewidth}{0.2mm}}
\noindent\begin{minipage}{\linewidth}
\medskip
\subsection{Query your database using plain English, fully on-premises}
\textsc{\footnotesize
{\scriptsize\faGlobe}\space 
vizly.fyi 
{\scriptsize\faThumbsOUp}\space 
\href{http://news.ycombinator.com/item?id=37315667\&utm\_term=comment}{101} 
{\scriptsize\faComments}\space 
\href{http://news.ycombinator.com/item?id=37315667\&utm\_term=comment}{17} 
}
\par\medskip\noindent
\href{https://www.vizly.fyi/?utm\_source=hackernewsletter\&utm\_medium=email\&utm\_term=data}{
    \includegraphics[width=0.99\linewidth]{35.png}
}
\end{minipage}
\paragraph{}
\textbf{plain English
Vizly powers instant, AI-powered insights for the enterprise. Runs fully on-premises, with no data leaving your secure network.}
\paragraph{}

Vizly Desktop is currently available for machines equipped with Apple Silicon.
Vizly leverages AI to transform your questions into actionable insights.
Query, filter, and visualize your data just by asking in plain English.
Generate insights with a click of a button to see instant analyses.
Snowflake
PostgreSQL
MySQL
Impala
Vizly runs completely on-device with no network calls to hosted APIs or services. This means all your data is safe on your own local network.
You can also securely share Vizly to anyone on the same network. Just copy and send the share link to anyone on the same network and they will be immediately able to run AI-powered queries, hosted from your device.
Vizly provides a wide range of visualization options to help you understand your data.
Change titles, colors, and charts just by asking. (Coming soon!)
When you're ready to go, export your charts and data in one click.
\dots\par
%\par\noindent\textcolor{red}{\rule{\linewidth}{0.2mm}}
\noindent\begin{minipage}{\linewidth}
\medskip
\subsection{Home – Database of Databases}
\textsc{\footnotesize
{\scriptsize\faCalendar}\space 
2023-01-01 
{\scriptsize\faGlobe}\space 
dbdb.io 
{\scriptsize\faThumbsOUp}\space 
\href{http://news.ycombinator.com/item?id=37314622\&utm\_term=comment}{87} 
{\scriptsize\faComments}\space 
\href{http://news.ycombinator.com/item?id=37314622\&utm\_term=comment}{8} 
}
\par\medskip\noindent
\href{https://dbdb.io/?utm\_source=hackernewsletter\&utm\_medium=email\&utm\_term=data}{
    \includegraphics[width=0.99\linewidth]{36.png}
}
\end{minipage}
\paragraph{}
Database of Databases
Browse
Leaderboards
Recent
Accounts
Login
Database of Databases
Discover and learn about
924
database management systems
Browse
Leaderboards
Most Recent
MarkLogic
iBoxDB
evitaDB
FalkorDB
SlyceIO
Most Viewed
BoltDB
Neon
LevelDB
NeDB
BTDB
Most Edited
Neon
Oxla
AgensGraph
BitYota
evitaDB
\dots\par
%\par\noindent\textcolor{red}{\rule{\linewidth}{0.2mm}}
\noindent\begin{minipage}{\linewidth}
\medskip
\subsection{Even Friendlier SQL with DuckDB}
\textsc{\footnotesize
{\scriptsize\faUser}\space 
Alex Monahan 
{\scriptsize\faCalendar}\space 
2023-08-23 
{\scriptsize\faGlobe}\space 
duckdb.org 
{\scriptsize\faThumbsOUp}\space 
\href{http://news.ycombinator.com/item?id=37262493\&utm\_term=comment}{30} 
{\scriptsize\faComments}\space 
\href{http://news.ycombinator.com/item?id=37262493\&utm\_term=comment}{None} 
}
\par\medskip\noindent
\href{https://duckdb.org/2023/08/23/even-friendlier-sql.html?utm\_source=hackernewsletter\&utm\_medium=email\&utm\_term=data}{
    \includegraphics[width=0.99\linewidth]{37.png}
}
\end{minipage}
\paragraph{}
\textbf{Even Friendlier SQL with DuckDB
TLDR; DuckDB continues to push the boundaries of SQL syntax to both simplify queries and make more advanced analyses possible.}
\paragraph{}
 Highlights include dynamic column selection, queries that start with the FROM clause, function chaining, and list comprehensions. We boldly go where no SQL engine has gone before!
Who says that SQL should stay frozen in time, chained to a 1999 version of the specification? As a comparison, do folks remember what JavaScript felt like before Promises? Those didn’t launch until 2012! It’s clear that innovation at the programming syntax layer can have a profoundly positive impact on an entire language ecosystem.
We believe there are many valid reasons for innovation in the SQL language, among them opportunities to simplify basic queries and also to make more dynamic analyses possible. Many of these features arose from community suggestions! Please let us know your SQL pain points on Discord or GitHub and join us as we change what it feels like to write SQL!
If you have not had a chance to read the first installment in this series, 
\dots\par
%\par\noindent\textcolor{red}{\rule{\linewidth}{0.2mm}}
\end{multicols}

\newpage
\section{\#Design}

\begin{multicols}{2}
\raggedcolumns
\noindent\begin{minipage}{\linewidth}
\medskip
\subsection{Where do fonts come from? Monotype, mostly}
\textsc{\footnotesize
{\scriptsize\faUser}\space 
Sara Friedman 
{\scriptsize\faCalendar}\space 
2023-08-25 
{\scriptsize\faGlobe}\space 
thehustle.co 
{\scriptsize\faThumbsOUp}\space 
\href{http://news.ycombinator.com/item?id=37283860\&utm\_term=comment}{326} 
{\scriptsize\faComments}\space 
\href{http://news.ycombinator.com/item?id=37283860\&utm\_term=comment}{31} 
}
\par\medskip\noindent
\href{https://thehustle.co/where-do-fonts-come-from/?utm\_source=hackernewsletter\&utm\_medium=email\&utm\_term=design}{
    \includegraphics[width=0.99\linewidth]{38.png}
}
\end{minipage}
\paragraph{}
\textbf{Ten years ago, Cindy Thomason was walking down the stairs at home when she heard her phone ring.
On the other end was an executive from Warner Bros.}
\paragraph{}
 Entertainment, calling to let her know that a font she designed would be featured in the upcoming blockbuster adaptation of The Great Gatsby.
“I had to sit down,” Thomason says. “I’m just somebody who decided to design a font on a whim.”
A nurse in suburban Virginia, Thomason began tinkering with fonts in her free time using a software package she bought for \$100. She’d listed the font, which she named Grandhappy, on an online marketplace called MyFonts.
That’s where producers from Warner Bros. found it, and bought it to use as Jay Gatsby’s handwriting in the 2013 film.
It should have been a dream come true, a big break for a hobbyist font designer. But Thomason’s cut for her design’s feature-film cameo was a whopping \$12 — not even enough to recoup what she paid for her design software.
Alternate letters designed by Thomason for her Grandhappy font (Cindy Thomason)
Thomason’s story isn’t an anomaly: Fonts are a ubiquitous commodity. Eve
\dots\par
%\par\noindent\textcolor{red}{\rule{\linewidth}{0.2mm}}
\noindent\begin{minipage}{\linewidth}
\medskip
\subsection{Ask HN: How to handle Asian-style “Family name first” when designing interfaces?}
\textsc{\footnotesize
{\scriptsize\faCalendar}\space 
2023-08-31 
{\scriptsize\faThumbsOUp}\space 
\href{}{256} 
{\scriptsize\faComments}\space 
\href{}{69} 
}
\par\medskip\noindent
\href{https://news.ycombinator.com/item?id=37332126\&utm\_source=hackernewsletter\&utm\_medium=email\&utm\_term=design}{
    \includegraphics[width=0.99\linewidth]{39.png}
}
\end{minipage}
\paragraph{}
\textbf{Our app is designed to be used across the Asia Pacific.
We have members who follow western naming conventions as well as members following common asian naming conventions.}
\paragraph{}

Turns out there can be alot of variation on what is the convention.
https://www.asiamediacentre.org.nz/features/a-guide-to-using...
How would you handle different naming conventions, so users see their name in the order they would like?
Family, Given
Given, Family
Indian name: Sathiavelllu Arunachalam, known as SA or Seth
SE Asian ethnic Chinese names: Harry Lee Kuan Yew, (English name) (Surname) (Given name). Hated the name Harry and got it removed, though many Chinese are referred to by an English name.
Indonesian name: Fatimah Azzahra (given name only)
Malaysian name: Sharifah Azizah binti Syed Ahmad Tarmizi, (honorific surname: Sharifah) (given name: Azizah) (patronym) (father's honorific surname: Syed) (father's given name: Ahmad Tarmizi)
reply
\dots\par
%\par\noindent\textcolor{red}{\rule{\linewidth}{0.2mm}}
\noindent\begin{minipage}{\linewidth}
\medskip
\subsection{UX design patterns for loading}
\textsc{\footnotesize
{\scriptsize\faUser}\space 
Pencil; Paper 
{\scriptsize\faCalendar}\space 
2022-05-24 
{\scriptsize\faGlobe}\space 
pencilandpaper.io 
{\scriptsize\faThumbsOUp}\space 
\href{http://news.ycombinator.com/item?id=37274610\&utm\_term=comment}{140} 
{\scriptsize\faComments}\space 
\href{http://news.ycombinator.com/item?id=37274610\&utm\_term=comment}{19} 
}
\par\medskip\noindent
\href{https://pencilandpaper.io/articles/ux-pattern-analysis-loading-feedback/?utm\_source=hackernewsletter\&utm\_medium=email\&utm\_term=design}{
    \includegraphics[width=0.99\linewidth]{40.png}
}
\end{minipage}
\paragraph{}
\textbf{On the Internet, loading is inevitable.}
\paragraph{}
 There are so many factors at play that you may have the fastest software out there, there’s always a chance that your user will find themselves waiting in front of a screen.
In dreamworld, online systems have absolutely zero delays, everything is snappy as can be. (Imagine!) But in the real world, when loading happens, you get the chance to reduce the perceived wait time for your user and increase the perceived value of your system.
No longer can you get away with just slapping a loading spinner on the page, and hoping the user will not table flip. This article is meant to help designers, developers, QAs and product people consider both the user’s and the system’s context before building the appropriate loading pattern.
Controlling the perception of time
Loading states are just that — small pockets of time that should be judiciously used to provide the user with visibility on what the system is doing. This is key in helping users feel in control, making them aware of the context at all times, and reassuring them that the right task is effective
\dots\par
%\par\noindent\textcolor{red}{\rule{\linewidth}{0.2mm}}
\end{multicols}

\newpage
\section{\#Books}

\begin{multicols}{2}
\raggedcolumns
\noindent\begin{minipage}{\linewidth}
\medskip
\subsection{Alexandria: A minimalistic cross-platform eBook reader}
\textsc{\footnotesize
{\scriptsize\faUser}\space 
Btpf 
{\scriptsize\faCalendar}\space 
2023-09-01 
{\scriptsize\faGithub}\space 
github.com 
{\scriptsize\faThumbsOUp}\space 
\href{http://news.ycombinator.com/item?id=37303960\&utm\_term=comment}{275} 
{\scriptsize\faComments}\space 
\href{http://news.ycombinator.com/item?id=37303960\&utm\_term=comment}{26} 
}
\par\medskip\noindent
\href{https://github.com/btpf/Alexandria?utm\_source=hackernewsletter\&utm\_medium=email\&utm\_term=books}{
    \includegraphics[width=0.99\linewidth]{41.png}
}
\end{minipage}
\paragraph{}
\textbf{Alexandria
Main Features:
= Completed = Work in Progress (For 1.0) = Planned (After 1.}
\paragraph{}
0)
- Supported Systems:
- Supported Formats:
- Custom Theme Support + Editor
- Custom Font + Font Downloader
- Highlights, Notes, and Bookmarks
- Reading progress slider with chapter marks
- Single-column, two-column, or continuous scrolling layouts
- Adjust Word Spacing, Line Height, and Reader Margins
- Highlight Exports
- Split Screen Layout
- Offline Dictionary Support
- Cross Platform Syncing
Screenshots:
Default Light \& Dark Themes
Annotations Popup
Theme Customizer
Reader Settings
Built With Node Version: v16.16.0
Credits
johnfactotum/foliate - Inspiration for building project, code snippets, and FictionBook + Comicbook Support
futurepress/epub.js - Providing foundation for project
bfabiszewski/libmobi - Providing Kindle Format Support
\dots\par
%\par\noindent\textcolor{red}{\rule{\linewidth}{0.2mm}}
\noindent\begin{minipage}{\linewidth}
\medskip
\subsection{North Korean science fiction}
\textsc{\footnotesize
{\scriptsize\faUser}\space 
Andrada Fiscutean 
{\scriptsize\faCalendar}\space 
2023-08-25 
{\scriptsize\faGlobe}\space 
arstechnica.com 
{\scriptsize\faThumbsOUp}\space 
\href{http://news.ycombinator.com/item?id=37291007\&utm\_term=comment}{192} 
{\scriptsize\faComments}\space 
\href{http://news.ycombinator.com/item?id=37291007\&utm\_term=comment}{7} 
}
\par\medskip\noindent
\href{https://arstechnica.com/culture/2023/08/the-strange-secretive-world-of-north-korean-science-fiction/?utm\_source=hackernewsletter\&utm\_medium=email\&utm\_term=books}{
    \includegraphics[width=0.99\linewidth]{42.png}
}
\end{minipage}
\paragraph{}
\textbf{A plane is flying to the Philippines, gliding above "the infinite surface" of the Pacific Ocean. Suddenly, a few passengers start to scream.}
\paragraph{}
 Soon, the captain announces there's a bomb on board, and it’s set to detonate if the aircraft drops below 10,000 feet.
"The inside of the plane turned into a battlefield," the story reads. "The captain was visibly startled and vainly tried to calm down the screaming and utterly terrorized passengers."
Only one person keeps his cool: a young North Korean diplomat who has faith that his country will find a solution and save everyone. And he’s right. North Korea's esteemed scientists and engineers create a mysterious anti-gravitational field and stop the plane in mid-air. The bomb is defused, and everyone gets off the aircraft and is brought back safely to Earth.
This story, Change Course (Hangno rǔl pakkura) by Yi Kŭmchǒl, speaks about solidarity, peace, and love for the motherland, displaying an intricate relationship between literature and politics. It was first published in 2004 in the Chosǒn munhak magazine, only to be reprinted 13 years later
\dots\par
%\par\noindent\textcolor{red}{\rule{\linewidth}{0.2mm}}
\noindent\begin{minipage}{\linewidth}
\medskip
\subsection{“Ansible for DevOps” eBook by Jeff Geerling Is Now Free}
\textsc{\footnotesize
{\scriptsize\faUser}\space 
Jeff Geerling 
{\scriptsize\faCalendar}\space 
2014-01-09 
{\scriptsize\faGlobe}\space 
leanpub.next 
{\scriptsize\faThumbsOUp}\space 
\href{http://news.ycombinator.com/item?id=37331212\&utm\_term=comment}{77} 
{\scriptsize\faComments}\space 
\href{http://news.ycombinator.com/item?id=37331212\&utm\_term=comment}{4} 
}
\par\medskip\noindent
\href{https://leanpub.com/ansible-for-devops/c/CTVMPCbEeXd3?utm\_source=hackernewsletter\&utm\_medium=email\&utm\_term=books}{
    \includegraphics[width=0.99\linewidth]{43.png}
}
\end{minipage}
\paragraph{}
Ansible for DevOps
Ansible for DevOps
Server and configuration management for humans
About the Book
Ansible is a simple, but powerful, server and configuration management tool (with a few other tricks up its sleeve). This book helps those familiar with the command line and basic shell scripting start using Ansible to provision and manage anywhere from one to thousands of servers.
The book begins with fundamentals, like installing Ansible, setting up a basic inventory file, and basic concepts, then guides you through Ansible's many uses, including ad-hoc commands, basic and advanced playbooks, application deployments, multiple-provider server provisioning, and even Docker orchestration! Everything is explained with pertinent real-world examples, often using Vagrant-managed virtual machines.
Examples in the book are tested with the latest version of Ansible.
Bundles that include this book
Table of Contents
- Foreword
-
Preface
- Second Edition
- Who is this book for?
- Typographic conventions
-
Please help improve this book!
- Current Published Book Version Information
- About the Auth
\dots\par
%\par\noindent\textcolor{red}{\rule{\linewidth}{0.2mm}}
\noindent\begin{minipage}{\linewidth}
\medskip
\subsection{Ask HN: Best books to learn web development?}
\textsc{\footnotesize
{\scriptsize\faCalendar}\space 
2023-08-30 
{\scriptsize\faThumbsOUp}\space 
\href{}{63} 
{\scriptsize\faComments}\space 
\href{}{20} 
}
\par\medskip\noindent
\href{https://news.ycombinator.com/item?id=37325594\&utm\_source=hackernewsletter\&utm\_medium=email\&utm\_term=books}{
    \includegraphics[width=0.99\linewidth]{44.png}
}
\end{minipage}
\paragraph{}
\textbf{Hi, I'm currently learning HTML, CSS and JS but I'd like to get more in depth.}
\paragraph{}
 I love learning stuff from books, even if it's a bit old-fashioned, because I can always take a brief look at what I need rather than having to search for basic information online, among several crappy SEO websites.
Can you suggest some books? I'd prefer advanced ones, as I'm not a newbie.
PS: I know you learn things only by doing, the book would be a support. You can suggest also books related to networking, that's the next step :)
[0]: https://alistapart.com/article/css-positioning-101/
For JS there are some online books that are gold mines the problem is finding them. One is Eloquent JavaScript which is available to read online for free here [1].
[1]: https://eloquentjavascript.net/
Though I've never actually finished the whole thing it does a pretty great job explaining JS to the uninitiated.
My other favorite books in JS are actually all from the same person Axel Rauschmayer. All of his books are available online for free here [2]. He has a really good way of explaining in my opinion and his books kin
\dots\par
%\par\noindent\textcolor{red}{\rule{\linewidth}{0.2mm}}
\noindent\begin{minipage}{\linewidth}
\medskip
\subsection{Library of Ashurbanipal}
\textsc{\footnotesize
{\scriptsize\faUser}\space 
Joshua J Mark 
{\scriptsize\faCalendar}\space 
2023-01-23 
{\scriptsize\faGlobe}\space 
worldhistory.org 
{\scriptsize\faThumbsOUp}\space 
\href{http://news.ycombinator.com/item?id=37333299\&utm\_term=comment}{49} 
{\scriptsize\faComments}\space 
\href{http://news.ycombinator.com/item?id=37333299\&utm\_term=comment}{9} 
}
\par\medskip\noindent
\href{https://www.worldhistory.org/Library\_of\_Ashurbanipal/?utm\_source=hackernewsletter\&utm\_medium=email\&utm\_term=books}{
    \includegraphics[width=0.99\linewidth]{45.png}
}
\end{minipage}
\paragraph{}
\textbf{The Library of Ashurbanipal (7th century BCE) is the oldest known systematically organized library in the world, established in Nineveh by the Neo-Assyrian king Ashurbanipal (r.}
\paragraph{}
 668-627 BCE) to preserve the history and culture of Mesopotamia. Over 30,000 texts were discovered at Nineveh in the mid-19th century, but the original collection is thought to have been much larger.
Contrary to often-repeated claims, the Library of Ashurbanipal was not the first library in the world. Libraries existed in Sumer, attached to scribal houses, temples, and palaces by the Early Dynastic Period (2900-2334 BCE). Akkadians and Babylonians also had libraries and so did earlier Assyrian kings. Scribes in ancient Mesopotamia also kept private libraries aside from those they would have referenced at the palace, school, or temple. The Library of Ashurbanipal is just the oldest one systematically organized to preserve a comprehensive collection of knowledge (not limited to one subject or type of work) and, owing to the importance of the tablets found there, the most significant. Scholar Paul Kriwaczek writ
\dots\par
%\par\noindent\textcolor{red}{\rule{\linewidth}{0.2mm}}
\noindent\begin{minipage}{\linewidth}
\medskip
\subsection{I built a website that lets you read classic books as email newsletters}
\textsc{\footnotesize
{\scriptsize\faCalendar}\space 
2023-01-01 
{\scriptsize\faGlobe}\space 
modernserial.com 
{\scriptsize\faThumbsOUp}\space 
\href{http://news.ycombinator.com/item?id=37296202\&utm\_term=comment}{37} 
{\scriptsize\faComments}\space 
\href{http://news.ycombinator.com/item?id=37296202\&utm\_term=comment}{4} 
}
\par\medskip\noindent
\href{https://modernserial.com/?utm\_source=hackernewsletter\&utm\_medium=email\&utm\_term=books}{
    \includegraphics[width=0.99\linewidth]{46.png}
}
\end{minipage}
\paragraph{}
\textbf{Here's how you find time to read the classics
Read the greatest books of all time as email newsletters in 10 minutes or less per day.}
\paragraph{}

Testimonials
What our readers are saying
Features
A newsletter just for you
Almost everything about your newsletter can be tweaked on the fly so that you can always find time to read, no matter how busy you get.
- Send it to me now.
- Reach the end of an email and want to keep going? Click one button and read the next one immediately.
- Pause anytime.
- When you're going on vacation or hit a busy period at work, it's simple to pause your emails until you're ready to start reading again.
- Tons of delivery options.
- Whether you want emails daily, once a week, or anything in between, we've got you covered. And if you ever need to rewind or fast-forward, you have complete control over what issue you'll get next.
Frequently asked questions
- How much does it cost?
-
We charge a flat fee of \$7.99 per book. Once you buy a book from us, it's yours forever. You can read it as many times as you want.
- Once I buy a book, how often will you send me new issues?
\dots\par
%\par\noindent\textcolor{red}{\rule{\linewidth}{0.2mm}}
\noindent\begin{minipage}{\linewidth}
\medskip
\subsection{Hugo and Nebula Award Short Stories}
\textsc{\footnotesize
{\scriptsize\faCalendar}\space 
2013-02-20 
{\scriptsize\faThumbsOUp}\space 
\href{http://news.ycombinator.com/item?id=37314493\&utm\_term=comment}{3} 
{\scriptsize\faComments}\space 
\href{http://news.ycombinator.com/item?id=37314493\&utm\_term=comment}{None} 
}
\par\medskip\noindent
\href{https://litsifi.com/short\_stories/hugo?utm\_source=hackernewsletter\&utm\_medium=email\&utm\_term=books}{
    \includegraphics[width=0.99\linewidth]{47.png}
}
\end{minipage}
\paragraph{}
LitSiFi
Hugo Award Short Story Winners
Nebula Award Short Story Winners
Signup
Log In
Hugo Award Short Story Winners
2022
Where Oaken Hearts Do Gather
By: Sarah Pinkser
Published In: Uncanny Magazine
Read Story
Discussions
2021
Metal Like Blood in the Dark
By: Ursula Vernon
Published In: Uncanny Magazine
Read Story
Discussions
2020
As the Last I May Know
By: S. L. Huang
Published In: Tor.com
Read Story
Discussions
2019
A Witch's Guide to Escape: A Practical Compendium of Portal Fantasies
By: Alix E. Harrow
Published In: Apex Magazine
Read Story
Discussions
2018
Welcome to Your Authentic Indian Experience™
By: Rebecca Roanhorse
Published In: Apex Magazine
Read Story
Discussions
2017
Seasons of Glass and Iron
By: Amal El-Mohtar
Published In: The Starlit Wood: New Fairy Tales (Saga Press)
Read Story
Discussions
2016
Cat Pictures Please
By: Naomi Kritzer
Published In: Clarkesworld Magazine
Read Story
Discussions
2014
The Water That Falls on You from Nowhere
By: John Chu
Published In: Tor.com
Read Story
Discussions
2013
Mono no Aware
By: Ken Liu
Published In: The Future is Japanese (Viz M
\dots\par
%\par\noindent\textcolor{red}{\rule{\linewidth}{0.2mm}}
\end{multicols}

\newpage
\section{\#Working}

\begin{multicols}{2}
\raggedcolumns
\noindent\begin{minipage}{\linewidth}
\medskip
\subsection{When your coworker does great work, tell their manager}
\textsc{\footnotesize
{\scriptsize\faUser}\space 
Julia Evans 
{\scriptsize\faCalendar}\space 
2020-07-14 
{\scriptsize\faGlobe}\space 
jvns.ca 
{\scriptsize\faThumbsOUp}\space 
\href{http://news.ycombinator.com/item?id=37340010\&utm\_term=comment}{515} 
{\scriptsize\faComments}\space 
\href{http://news.ycombinator.com/item?id=37340010\&utm\_term=comment}{42} 
}
\par\medskip\noindent
\href{https://jvns.ca/blog/2020/07/14/when-your-coworker-does-great-work-tell-their-manager/?utm\_source=hackernewsletter\&utm\_medium=email\&utm\_term=working}{
    \includegraphics[width=0.99\linewidth]{48.png}
}
\end{minipage}
\paragraph{}
\textbf{I’ve been thinking recently about anti-racism and what it looks like to support colleagues from underrepresented groups at work.}
\paragraph{}
 The other day someone in a Slack group made an offhand comment that they’d sent a message to an engineer’s manager to say that the engineer was doing exceptional work.
I think telling someone’s manager they’re doing great work is a pretty common practice and it can be really helpful, but it’s easy to forget to do and I wish someone had suggested it to me earlier. So let’s talk about it!
I tweeted about this to ask how people approach it and as usual I got a ton of great replies that I’m going to summarize here.
We’re going to talk about what to say, when to do this, and why you should ask first.
ask if it’s ok first
One thing that at least 6 different people brought up was the importance of asking first. It might not be obvious why this is important at first — you’re saying something positive! What’s the problem?
So here are some potential reasons saying something positive to someone’s manager could backfire:
- Giving someone a compliment that’s not in line
\dots\par
%\par\noindent\textcolor{red}{\rule{\linewidth}{0.2mm}}
\noindent\begin{minipage}{\linewidth}
\medskip
\subsection{Can a worker-owned restaurant work?}
\textsc{\footnotesize
{\scriptsize\faUser}\space 
Editor 
{\scriptsize\faCalendar}\space 
2023-08-26 
{\scriptsize\faGlobe}\space 
southseattleemerald.com 
{\scriptsize\faThumbsOUp}\space 
\href{http://news.ycombinator.com/item?id=37279895\&utm\_term=comment}{231} 
{\scriptsize\faComments}\space 
\href{http://news.ycombinator.com/item?id=37279895\&utm\_term=comment}{64} 
}
\par\medskip\noindent
\href{https://southseattleemerald.com/2023/08/26/can-a-worker-owned-restaurant-work/?utm\_source=hackernewsletter\&utm\_medium=email\&utm\_term=working}{
    \includegraphics[width=0.99\linewidth]{49.png}
}
\end{minipage}
\paragraph{}
\textbf{by Tobias Coughlin-Bogue
(This article was originally published on Real Change and has been reprinted under an agreement.)
Restaurants run on hierarchy, or so I’ve always been told.}
\paragraph{}
 There’s got to be someone in charge, someone giving orders, in order for the whole thing to run right. Whatever situation you find restaurant work analogous to, be it the military, a sports team, or an orchestra, there’s someone calling the shots.
The last person I worked for, one of the most experienced and talented restaurant people I’ve ever met, always said it’s best to run a restaurant as a “benign dictatorship.” Under her leadership, which was very much from the front, that didn’t seem so bad. She’d done every dirty job in the place and didn’t hesitate to do them every night alongside the rest of us.
But that’s not every restaurant owner.
I’ve worked at plenty of places where the owner was just in the way. Or worse, as was the case of the owner of a “fine casual” Vietnamese restaurant I worked at. Notorious for bawling out his employees, he’d sit at a corner table every night, drinking whatever red 
\dots\par
%\par\noindent\textcolor{red}{\rule{\linewidth}{0.2mm}}
\noindent\begin{minipage}{\linewidth}
\medskip
\subsection{Raise less, build more}
\textsc{\footnotesize
{\scriptsize\faUser}\space 
The Author trohan 
{\scriptsize\faCalendar}\space 
2023-08-20 
{\scriptsize\faGlobe}\space 
trohan.com 
{\scriptsize\faThumbsOUp}\space 
\href{http://news.ycombinator.com/item?id=37298665\&utm\_term=comment}{165} 
{\scriptsize\faComments}\space 
\href{http://news.ycombinator.com/item?id=37298665\&utm\_term=comment}{16} 
}
\par\medskip\noindent
\href{https://trohan.com/2023/08/20/raise-less-build-more/?utm\_source=hackernewsletter\&utm\_medium=email\&utm\_term=working}{
    \includegraphics[width=0.99\linewidth]{50.png}
}
\end{minipage}
\paragraph{}
\textbf{The conventional venture capital funding path – from raising an institutional Seed, Series A, B, C, D, E, etc, all the way to exit via IPO – has long been treated as gospel.}
\paragraph{}
 Its verses are most heavily preached by VC board members, whose business model it also supports.
But there is an influential tide of founders on the rise that is opting out of this path and quietly plotting a new one that leads to building generational companies.
It’s a hybrid path, combining the growth of targeted venture funding with the durability found in bootstrapping (i.e. profitability). It’s a path with less venture capital and more self-reliance.
And it’s the direct result of founders emerging from a tumultuous period of feast (with 5x more venture capital offered to startups over the past decade) and a brief flirtation with famine from the recent pullback that has left some venture-dependent companies in starvation mode.
For many founders, a steady reliance on venture capital, as it is heavily prescribed today, is often seen as unhealthy, if not risky.
Increasingly, these founders are seeking freedom fr
\dots\par
%\par\noindent\textcolor{red}{\rule{\linewidth}{0.2mm}}
\noindent\begin{minipage}{\linewidth}
\medskip
\subsection{We always end up with waterfall}
\textsc{\footnotesize
{\scriptsize\faCalendar}\space 
2000-01-01 
{\scriptsize\faThumbsOUp}\space 
\href{http://news.ycombinator.com/item?id=37305698\&utm\_term=comment}{148} 
{\scriptsize\faComments}\space 
\href{http://news.ycombinator.com/item?id=37305698\&utm\_term=comment}{29} 
}
\par\medskip\noindent
\href{https://www.amazingcto.com/why-we-always-endup-with-waterfall-even-scrum/?utm\_source=hackernewsletter\&utm\_medium=email\&utm\_term=working}{
    \includegraphics[width=0.99\linewidth]{51.png}
}
\end{minipage}
\paragraph{}
\textbf{A coachee recently wanted to make their process less waterfally. He wanted to talk about a new process that was less waterfall than their current one - they were doing Scrum.}
\paragraph{}

This made me wonder why we always end up with waterfall and how to fix it. I realized a new process will not be sufficient. We got Scrum and ended up with Scrumfall. It is not the process that brings waterfall, but drivers beyond the process that forge our environment. How does that work?
We had elaborated waterfall processes like Rational Unified Process (RUP) in the 90s, mainly because people believed software development was about engineering. So, they copied processes from engineering machines or buildings. Other driving forces were the focus on correctness at the first time (because of difficult releases and long release cycles) and predictability.
💬 Contact
You're a CTO and think about coaching? Let's talk!
Then the internet arrived, and things speed up. The market and competitors changed faster as ever. Innovation brought an explosion of new business and product ideas. With waterfall, we were building thi
\dots\par
%\par\noindent\textcolor{red}{\rule{\linewidth}{0.2mm}}
\noindent\begin{minipage}{\linewidth}
\medskip
\subsection{Earning the privilege to work on unoriginal problems}
\textsc{\footnotesize
{\scriptsize\faUser}\space 
Marcus Kohlberg 
{\scriptsize\faCalendar}\space 
2023-08-24 
{\scriptsize\faGlobe}\space 
substack.com 
{\scriptsize\faThumbsOUp}\space 
\href{http://news.ycombinator.com/item?id=37272362\&utm\_term=comment}{113} 
{\scriptsize\faComments}\space 
\href{http://news.ycombinator.com/item?id=37272362\&utm\_term=comment}{18} 
}
\par\medskip\noindent
\href{https://landmines.substack.com/p/earning-the-privilege-to-work-on?utm\_source=hackernewsletter\&utm\_medium=email\&utm\_term=working}{
    \includegraphics[width=0.99\linewidth]{52.png}
}
\end{minipage}
\paragraph{}
\textbf{Earning the privilege to work on unoriginal problems
The startup world is almost by definition a place where you “live in the future”.}
\paragraph{}
 Working in an early-stage startup, your whole belief system is based on being able to successfully create something new. Something that will become a great and cherished product, used all around the world. It needs to be this way, or else you won’t be able to cope with the stress and insecurity of working in a startup.
What’s particularly dangerous though, is that from this coping mechanism of blind belief springs a deceptive rationality of investing in systems and tools that will support this grand future. This is why so many startup dev teams are hellbent on frontloading an investment in building the perfect dev-tooling, and ensuring the infrastructure setup is highly flexible and scalable. After all, since your product is going to take over the world, you must get ready for scale. Right? Wrong.
You need to earn the privilege of working on unoriginal problems
Let’s face it, in this day and age, scaling software is not an original problem. It’s been 
\dots\par
%\par\noindent\textcolor{red}{\rule{\linewidth}{0.2mm}}
\end{multicols}

\newpage
\section{\#Learn}

\begin{multicols}{2}
\raggedcolumns
\noindent\begin{minipage}{\linewidth}
\medskip
\subsection{A DIY ‘bionic pancreas’ is changing diabetes care}
\textsc{\footnotesize
{\scriptsize\faUser}\space 
Drew; Liam 
{\scriptsize\faCalendar}\space 
2023-08-30 
{\scriptsize\faGlobe}\space 
nature.com 
{\scriptsize\faThumbsOUp}\space 
\href{http://news.ycombinator.com/item?id=37321028\&utm\_term=comment}{452} 
{\scriptsize\faComments}\space 
\href{http://news.ycombinator.com/item?id=37321028\&utm\_term=comment}{31} 
}
\par\medskip\noindent
\href{https://www.nature.com/articles/d41586-023-02648-9?utm\_source=hackernewsletter\&utm\_medium=email\&utm\_term=learn}{
    \includegraphics[width=0.99\linewidth]{53.png}
}
\end{minipage}
\paragraph{}
\textbf{Ten years ago, a tech-savvy group of people with type 1 diabetes (T1D) decided to pursue a DIY approach to their own treatment.}
\paragraph{}
 They knew that a fairly straightforward piece of software could make their lives much easier, but no companies were developing it quickly enough.
What this software promised was freedom from having to constantly measure and control their blood-glucose levels. In people without T1D, when glucose levels rise, cells in the pancreas release insulin, a hormone that helps tissues to absorb that glucose. In T1D, these cells are killed by the immune system, leaving people with the condition to manage their blood sugar by taking insulin.
“It is almost inhumane,” says Shane O’Donnell, a medical sociologist at University College Dublin, who, like everyone quoted in this article, lives with T1D. “You’re constantly having to think about diabetes in order to survive.”
Members of the nascent DIY community were using the most sophisticated technology available: insulin pumps and wearable devices called constant glucose monitors. But they still had to read the monitor’s data
\dots\par
%\par\noindent\textcolor{red}{\rule{\linewidth}{0.2mm}}
\noindent\begin{minipage}{\linewidth}
\medskip
\subsection{A new method to reprogram human cells to better mimic embryonic stem cells}
\textsc{\footnotesize
{\scriptsize\faCalendar}\space 
2023-08-17 
{\scriptsize\faGlobe}\space 
uwa.edu.au 
{\scriptsize\faThumbsOUp}\space 
\href{http://news.ycombinator.com/item?id=37286860\&utm\_term=comment}{406} 
{\scriptsize\faComments}\space 
\href{http://news.ycombinator.com/item?id=37286860\&utm\_term=comment}{7} 
}
\par\medskip\noindent
\href{https://www.uwa.edu.au/news/Article/2023/August/Scientists-find-way-to-wipe-a-cells-memory-to-reprogram-it-as-a-stem-cell?utm\_source=hackernewsletter\&utm\_medium=email\&utm\_term=learn}{
    \includegraphics[width=0.99\linewidth]{54.png}
}
\end{minipage}
\paragraph{}
\textbf{In a groundbreaking study published today in Nature, Australian scientists have resolved a long-standing problem in regenerative medicine.}
\paragraph{}
 Led by Professor Ryan Lister from the Harry Perkins Institute of Medical Research and The University of Western Australia and Professor Jose M Polo from Monash University and the University of Adelaide, the team developed a new method to reprogram human cells to better mimic embryonic stem cells, with significant implications for biomedical and therapeutic uses.
“We predict that TNT reprogramming will establish a new benchmark for cell therapies and biomedical research, and substantially advance their progress.”Professor Ryan Lister, UWA Centre for Medical Research
In a revolutionary advance in the mid-2000s, it was discovered that the non-reproductive adult cells of the body, called ‘somatic’ cells, could be artificially reprogrammed into a state that resembles embryonic stem (ES) cells which have the capacity to then generate any cell of the body.
The ability to artificially reprogram human somatic cells, such as skin cells, into these so-called
\dots\par
%\par\noindent\textcolor{red}{\rule{\linewidth}{0.2mm}}
\noindent\begin{minipage}{\linewidth}
\medskip
\subsection{How to drill your own water well}
\textsc{\footnotesize
{\scriptsize\faCalendar}\space 
2023-09-01 
{\scriptsize\faGlobe}\space 
drillyourownwell.com 
{\scriptsize\faThumbsOUp}\space 
\href{http://news.ycombinator.com/item?id=37257514\&utm\_term=comment}{338} 
{\scriptsize\faComments}\space 
\href{http://news.ycombinator.com/item?id=37257514\&utm\_term=comment}{30} 
}
\par\medskip\noindent
\href{https://drillyourownwell.com/?utm\_source=hackernewsletter\&utm\_medium=email\&utm\_term=learn}{
    \includegraphics[width=0.99\linewidth]{55.png}
}
\end{minipage}
\paragraph{}
\textbf{85 web pages and 52 videos entirely devoted
to helping you drill your own well
You can drill your own shallow water well using PVC and household water hoses.}
\paragraph{}
 It is a cheap and effective way to dig your own shallow water well. Water well drilling isn’t just for the pros with huge commercial drilling rigs. Digging a water well yourself is both interesting and fun.
The water well drilling methods described here work well in digging/drilling through dirt, and clay, including really hard clay. They will not work if you need to drill through rock but, if the area you live in is flat or relatively flat, it is definitely worth a try. Many folks think they have to dig or drill their well into an aquifer. For irrigation and lawn watering, reaching an aquifer isn’t necessary. You only have to drill under the standing water level. It is very likely that you can drill your own well. Many successful wells have been drilled using this well drilling method. It is cheap. You can expect the “drilling” portion of the project to cost about \$200.
YOU CAN DRILL YOUR OWN WELL
In these pages the “do it your
\dots\par
%\par\noindent\textcolor{red}{\rule{\linewidth}{0.2mm}}
\noindent\begin{minipage}{\linewidth}
\medskip
\subsection{111,111.1 meters is reliably 1 degree of latitude}
\textsc{\footnotesize
{\scriptsize\faUser}\space 
Thomas O Thomas O 1 
{\scriptsize\faCalendar}\space 
2016-07-26 
{\scriptsize\faGlobe}\space 
stackexchange.com 
{\scriptsize\faThumbsOUp}\space 
\href{http://news.ycombinator.com/item?id=37284487\&utm\_term=comment}{338} 
{\scriptsize\faComments}\space 
\href{http://news.ycombinator.com/item?id=37284487\&utm\_term=comment}{7} 
}
\par\medskip\noindent
\href{https://gis.stackexchange.com/a/2964/5599?utm\_source=hackernewsletter\&utm\_medium=email\&utm\_term=learn}{
    \includegraphics[width=0.99\linewidth]{56.png}
}
\end{minipage}
\paragraph{}
\textbf{I'm looking for an algorithm which when given a latitude and longitude pair and a vector translation in meters in Cartesian coordinates (x,y) would give me a new coordinate.}
\paragraph{}
 Sort of like a reverse Haversine. I could also work with a distance and a heading transformation, but this would probably be slower and not as accurate. Ideally, the algorithm should be fast as I'm working on an embedded system. Accuracy is not critical, within 10 meters would be good.
-
So you'd be fine modeling the earth as a sphere?– underdark ♦Oct 26, 2010 at 22:54
-
1Yeah, that would be fine as I'm expecting <1km offsets.– Thomas OOct 26, 2010 at 22:58
8 Answers
If your displacements aren't too great (less than a few kilometers) and you're not right at the poles, use the quick and dirty estimate that 111,111 meters (111.111 km) in the y direction is 1 degree (of latitude) and 111,111 * cos(latitude) meters in the x direction is 1 degree (of longitude).
-
6@Thomas: Actually, you can be very close to the poles. I checked against a UTM calculation using equal x- and y-displacements of 1400 m (so the total displ
\dots\par
%\par\noindent\textcolor{red}{\rule{\linewidth}{0.2mm}}
\noindent\begin{minipage}{\linewidth}
\medskip
\subsection{How far can you jump from a swing?}
\textsc{\footnotesize
{\scriptsize\faCalendar}\space 
2023-08-18 
{\scriptsize\faThumbsOUp}\space 
\href{http://news.ycombinator.com/item?id=37313493\&utm\_term=comment}{233} 
{\scriptsize\faComments}\space 
\href{http://news.ycombinator.com/item?id=37313493\&utm\_term=comment}{22} 
}
\par\medskip\noindent
\href{https://www.alexmolas.com/2023/08/18/how-far-can-you-jump.html?utm\_source=hackernewsletter\&utm\_medium=email\&utm\_term=learn}{
    \includegraphics[width=0.99\linewidth]{57.png}
}
\end{minipage}
\paragraph{}
\textbf{How far can you jump from a swing?
August 18, 2023
Discussion on HackerNews.}
\paragraph{}

Some people pointed out some flaws in my modelling (eg: assuming zero distance from swing to floor) which I’ve tried to fix. The original maximum distance estimation was around \$1m\$.
This summer I’ve spent an absurd amount of time reading and learning about the physics of swings. Yes, you read it right, I’ve been learning about the physical processes that happen when a kid is playing with a swing in the park. Blame it on my kids and the countless hours spent enjoying these moments with them. In particular, I read about the physics of pumping a swing and about the physics of jumping from a swing. Amidst my deep dive into swing physics, I came up with a new Olympic sport in which you start seated on a swing with length \$L\$, your feet comfortably touching the ground. As a countdown of \$T\$ seconds commences, you embark on the art of swing-pumping. Your challenge is to execute a skillful leap before the countdown reaches zero. With your jump, you travel a distance \$d\$ from your initial point, aiming to achieve th
\dots\par
%\par\noindent\textcolor{red}{\rule{\linewidth}{0.2mm}}
\end{multicols}

\newpage
\section{\#Watching}

\begin{multicols}{2}
\raggedcolumns
\noindent\begin{minipage}{\linewidth}
\medskip
\subsection{Slime Molds}
\textsc{\footnotesize
{\scriptsize\faCalendar}\space 
2022-01-03 
{\scriptsize\faYoutube}\space 
youtube.com 
{\scriptsize\faThumbsOUp}\space 
\href{http://news.ycombinator.com/item?id=37265664\&utm\_term=comment}{132} 
{\scriptsize\faComments}\space 
\href{http://news.ycombinator.com/item?id=37265664\&utm\_term=comment}{15} 
}
\par\medskip\noindent
\href{https://www.youtube.com/watch?v=gpt9cJrEZ\_Y\&utm\_source=hackernewsletter\&utm\_medium=email\&utm\_term=watching}{
    \includegraphics[width=0.99\linewidth]{58.jpg}
}
\end{minipage}
\paragraph{}
Over
Pers
Auteursrecht
Contact
Creators
Adverteren
Ontwikkelaars
Voorwaarden
Privacy
Beleid en veiligheid
Zo werkt YouTube
Nieuwe functies testen
© 2023 Google LLC
YouTube, een bedrijf van Google
\dots\par
%\par\noindent\textcolor{red}{\rule{\linewidth}{0.2mm}}
\noindent\begin{minipage}{\linewidth}
\medskip
\subsection{Electronics Course (45 episodes, YouTube)}
\textsc{\footnotesize
{\scriptsize\faCalendar}\space 
2023-08-31 
{\scriptsize\faThumbsOUp}\space 
\href{http://news.ycombinator.com/item?id=37339348\&utm\_term=comment}{131} 
{\scriptsize\faComments}\space 
\href{http://news.ycombinator.com/item?id=37339348\&utm\_term=comment}{6} 
}
\par\medskip\noindent
\href{https://www.youtube.com/playlist?list=PL7qUW0KPfsIIOPOKL84wK\_Qj9N7gvJX6v\&utm\_source=hackernewsletter\&utm\_medium=email\&utm\_term=watching}{
    \includegraphics[width=0.99\linewidth]{notfound.png}
}
\end{minipage}
\paragraph{}
Over
Pers
Auteursrecht
Contact
Creators
Adverteren
Ontwikkelaars
Voorwaarden
Privacy
Beleid en veiligheid
Zo werkt YouTube
Nieuwe functies testen
© 2023 Google LLC
YouTube, een bedrijf van Google
Razavi Electronics - YouTube
\dots\par
%\par\noindent\textcolor{red}{\rule{\linewidth}{0.2mm}}
\noindent\begin{minipage}{\linewidth}
\medskip
\subsection{Tokyo by Train}
\textsc{\footnotesize
{\scriptsize\faCalendar}\space 
2016-12-09 
{\scriptsize\faYoutube}\space 
youtube.com 
{\scriptsize\faThumbsOUp}\space 
\href{http://news.ycombinator.com/item?id=37276630\&utm\_term=comment}{96} 
{\scriptsize\faComments}\space 
\href{http://news.ycombinator.com/item?id=37276630\&utm\_term=comment}{4} 
}
\par\medskip\noindent
\href{https://www.youtube.com/watch?v=Y49VfddU-L4\&utm\_source=hackernewsletter\&utm\_medium=email\&utm\_term=watching}{
    \includegraphics[width=0.99\linewidth]{60.jpg}
}
\end{minipage}
\paragraph{}
Over
Pers
Auteursrecht
Contact
Creators
Adverteren
Ontwikkelaars
Voorwaarden
Privacy
Beleid en veiligheid
Zo werkt YouTube
Nieuwe functies testen
© 2023 Google LLC
YouTube, een bedrijf van Google
\dots\par
%\par\noindent\textcolor{red}{\rule{\linewidth}{0.2mm}}
\noindent\begin{minipage}{\linewidth}
\medskip
\subsection{Sal Khan: How AI could save (not destroy) education}
\textsc{\footnotesize
{\scriptsize\faCalendar}\space 
2023-05-01 
{\scriptsize\faYoutube}\space 
youtube.com 
{\scriptsize\faThumbsOUp}\space 
\href{http://news.ycombinator.com/item?id=37283191\&utm\_term=comment}{88} 
{\scriptsize\faComments}\space 
\href{http://news.ycombinator.com/item?id=37283191\&utm\_term=comment}{10} 
}
\par\medskip\noindent
\href{https://www.youtube.com/watch?v=hJP5GqnTrNo\&utm\_source=hackernewsletter\&utm\_medium=email\&utm\_term=watching}{
    \includegraphics[width=0.99\linewidth]{61.jpg}
}
\end{minipage}
\paragraph{}
Over
Pers
Auteursrecht
Contact
Creators
Adverteren
Ontwikkelaars
Voorwaarden
Privacy
Beleid en veiligheid
Zo werkt YouTube
Nieuwe functies testen
© 2023 Google LLC
YouTube, een bedrijf van Google
\dots\par
%\par\noindent\textcolor{red}{\rule{\linewidth}{0.2mm}}
\end{multicols}

\newpage
\section{\#Startup News}

\begin{multicols}{2}
\raggedcolumns
\noindent\begin{minipage}{\linewidth}
\medskip
\subsection{Amazon acquires Fig}
\textsc{\footnotesize
{\scriptsize\faUser}\space 
Brendan Falk 
{\scriptsize\faCalendar}\space 
2023-08-28 
{\scriptsize\faGlobe}\space 
fig.io 
{\scriptsize\faThumbsOUp}\space 
\href{http://news.ycombinator.com/item?id=37296401\&utm\_term=comment}{376} 
{\scriptsize\faComments}\space 
\href{http://news.ycombinator.com/item?id=37296401\&utm\_term=comment}{49} 
}
\par\medskip\noindent
\href{https://fig.io/blog/post/fig-joins-aws?utm\_source=hackernewsletter\&utm\_medium=email\&utm\_term=startup\_news}{
    \includegraphics[width=0.99\linewidth]{62.png}
}
\end{minipage}
\paragraph{}
\textbf{Fig has joined AWS!
I am thrilled to announce that the Fig team will be joining Amazon Web Services (AWS) and Amazon has acquired Fig's technology!}
\paragraph{}

Fig and AWS share a passion for improved developer tools and services. By combining Fig's expertise with AWS' long-term orientation and track record of delivering customer-centric products, we see an opportunity to enhance the developer experience. AWS believes that generative AI represents a major technological shift to transform the way its customers build, and we are beyond excited to be a part of that larger vision.
What's happening to Fig's product?
Existing users will continue to be able to use Fig and will receive ongoing support. What's more, we are now making all the paid Fig Team features completely free. New users will not be able to sign up for Fig's products right now while we focus on optimizing them for existing customers and addressing some needs identified to integrate Fig with AWS. If you're interested in staying up-to-date on our work with AWS, we will release a signup link in the coming weeks!
What products will Amazon
\dots\par
%\par\noindent\textcolor{red}{\rule{\linewidth}{0.2mm}}
\noindent\begin{minipage}{\linewidth}
\medskip
\subsection{Tesla braces for its first trial involving Autopilot fatality}
\textsc{\footnotesize
{\scriptsize\faUser}\space 
Dan Levine; Hyunjoo Jin 
{\scriptsize\faCalendar}\space 
2023-08-28 
{\scriptsize\faGlobe}\space 
reuters.com 
{\scriptsize\faThumbsOUp}\space 
\href{http://news.ycombinator.com/item?id=37293103\&utm\_term=comment}{145} 
{\scriptsize\faComments}\space 
\href{http://news.ycombinator.com/item?id=37293103\&utm\_term=comment}{21} 
}
\par\medskip\noindent
\href{https://www.reuters.com/business/autos-transportation/tesla-braces-its-first-trial-involving-autopilot-fatality-2023-08-28/?utm\_source=hackernewsletter\&utm\_medium=email\&utm\_term=startup\_news}{
    \includegraphics[width=0.99\linewidth]{63.png}
}
\end{minipage}
\paragraph{}
\textbf{Focus: Tesla braces for its first trial involving Autopilot fatality
SAN FRANCISCO, Aug 28 (Reuters) - Tesla Inc (TSLA.}
\paragraph{}
O) is set to defend itself for the first time at trial against allegations that failure of its Autopilot driver assistant feature led to death, in what will likely be a major test of Chief Executive Elon Musk's assertions about the technology.
Self-driving capability is central to Tesla’s financial future, according to Musk, whose own reputation as an engineering leader is being challenged with allegations by plaintiffs in one of two lawsuits that he personally leads the group behind technology that failed. Wins by Tesla could raise confidence and sales for the software, which costs up to \$15,000 per vehicle.
Tesla faces two trials in quick succession, with more to follow.
The first, scheduled for mid-September in a California state court, is a civil lawsuit containing allegations that the Autopilot system caused owner Micah Lee’s Model 3 to suddenly veer off a highway east of Los Angeles at 65 miles per hour, strike a palm tree and burst into flames, all in the span
\dots\par
%\par\noindent\textcolor{red}{\rule{\linewidth}{0.2mm}}
\noindent\begin{minipage}{\linewidth}
\medskip
\subsection{Amazon CEO reportedly told remote employees: It’s probably not going to work out}
\textsc{\footnotesize
{\scriptsize\faUser}\space 
Emma Roth 
{\scriptsize\faCalendar}\space 
2023-08-28 
{\scriptsize\faGlobe}\space 
theverge.com 
{\scriptsize\faThumbsOUp}\space 
\href{http://news.ycombinator.com/item?id=37301728\&utm\_term=comment}{110} 
{\scriptsize\faComments}\space 
\href{http://news.ycombinator.com/item?id=37301728\&utm\_term=comment}{49} 
}
\par\medskip\noindent
\href{https://www.theverge.com/2023/8/28/23849754/amazon-ceo-andy-jassy-remote-employees-return-to-office?utm\_source=hackernewsletter\&utm\_medium=email\&utm\_term=startup\_news}{
    \includegraphics[width=0.99\linewidth]{64.png}
}
\end{minipage}
\paragraph{}
\textbf{Amazon CEO Andy Jassy has a message to employees who don’t want to return to the office: “It’s not going to work out for you.}
\paragraph{}
” That’s according to a report from Insider, which says Jassy made that statement during a meeting earlier this month.
While Amazon ordered its employees to return to the office for three days per week starting in May, many Amazon employees weren’t happy about the decision. Thousands of workers signed a petition against the mandate and staged a walkout in response.
That clearly hasn’t changed Amazon’s position on the matter. In a recording of the meeting obtained by Insider, Jassy told workers, “It’s past the time to disagree and commit,” adding that “if you can’t disagree and commit... it’s probably not going to work out for you at Amazon because we are going back to the office at least three days a week.”
Jassy reportedly said his decision to have employees return to the office was a “judgment call” and that employees can leave if they don’t want to comply. “It’s not right for all of our teammates to be in three days a week and for people to refuse to do so,”
\dots\par
%\par\noindent\textcolor{red}{\rule{\linewidth}{0.2mm}}
\noindent\begin{minipage}{\linewidth}
\medskip
\subsection{Instacart S-1}
\textsc{\footnotesize
{\scriptsize\faThumbsOUp}\space 
\href{http://news.ycombinator.com/item?id=37265182\&utm\_term=comment}{76} 
{\scriptsize\faComments}\space 
\href{http://news.ycombinator.com/item?id=37265182\&utm\_term=comment}{4} 
}
\par\medskip\noindent
\href{https://www.sec.gov/Archives/edgar/data/1579091/000119312523221345/d55348ds1.htm?utm\_source=hackernewsletter\&utm\_medium=email\&utm\_term=startup\_news}{
    \includegraphics[width=0.99\linewidth]{65.png}
}
\end{minipage}
\paragraph{}

\dots\par
%\par\noindent\textcolor{red}{\rule{\linewidth}{0.2mm}}
\end{multicols}

\newpage
\section{\#Fun}

\begin{multicols}{2}
\raggedcolumns
\noindent\begin{minipage}{\linewidth}
\medskip
\subsection{Leaked Wipeout source code leads to near-total rewrite and remaster}
\textsc{\footnotesize
{\scriptsize\faUser}\space 
Kevin Purdy 
{\scriptsize\faCalendar}\space 
2023-08-24 
{\scriptsize\faGlobe}\space 
arstechnica.com 
{\scriptsize\faThumbsOUp}\space 
\href{http://news.ycombinator.com/item?id=37262258\&utm\_term=comment}{247} 
{\scriptsize\faComments}\space 
\href{http://news.ycombinator.com/item?id=37262258\&utm\_term=comment}{15} 
}
\par\medskip\noindent
\href{https://arstechnica.com/gaming/2023/08/developer-rewrites-original-wipeout-from-abysmal-leaked-windows-source/?utm\_source=hackernewsletter\&utm\_medium=email\&utm\_term=fun}{
    \includegraphics[width=0.99\linewidth]{66.png}
}
\end{minipage}
\paragraph{}
There have been a lot of Wipeout games released since the 1995 original, including Wipeout HD and the Omega Collection, but only the original has the distinction of having its Windows port source code leaked by (since defunct) archive Forest of Illusion.
Dominic Szablewski grabbed that code before it disappeared and set about creating a version that’s not just a port. He rewrote the game’s rendering, physics, sound, and generally “everything everywhere.” He documented the project, put his code on GitHub, and has some version of a justification. “So let's just pretend that the leak was intentional, a rewrite of the source falls under fair use and the whole thing is abandonware anyway,” Szablewski writes.
Most of the code seemed to come from Wipeout ATI 3D Rage Edition, a “lackluster port for Windows” that was bundled with ATI GPUs, Szablewski wrote. It is a mess. There are fragments of code versions from DOS, PlayStation, Windows 95, and Windows 98, with lots of things shakily patched in, including some kludgey 25-to-30 frames-per-second physics calculations in moving from European PA
\dots\par
%\par\noindent\textcolor{red}{\rule{\linewidth}{0.2mm}}
\noindent\begin{minipage}{\linewidth}
\medskip
\subsection{Linux on a Commodore 64}
\textsc{\footnotesize
{\scriptsize\faUser}\space 
Onnokort 
{\scriptsize\faCalendar}\space 
2023-08-31 
{\scriptsize\faGithub}\space 
github.com 
{\scriptsize\faThumbsOUp}\space 
\href{http://news.ycombinator.com/item?id=37277907\&utm\_term=comment}{226} 
{\scriptsize\faComments}\space 
\href{http://news.ycombinator.com/item?id=37277907\&utm\_term=comment}{20} 
}
\par\medskip\noindent
\href{https://github.com/onnokort/semu-c64?utm\_source=hackernewsletter\&utm\_medium=email\&utm\_term=fun}{
    \includegraphics[width=0.99\linewidth]{67.png}
}
\end{minipage}
\paragraph{}
\textbf{Running Linux on a Commodore C-64
"But does it run Linux?" can now be finally and affirmatively answered for the Commodore C64!}
\paragraph{}

There is a catch (rather: a couple) of course: It runs extremely slowly and it needs a RAM Expansion Unit (REU), as there is no chance to fit it all into just 64KiB.
It even emulates virtual memory with an MMU.
ChangeLog / Updates
Aug 31th 2023, Persistence (quickly load a booted Linux)
I added simple persistence to the emulator so it is possible to "quickly" (ahem) run Linux on your C-64.
The emulator can load the RISC-V CPU registers (plus the other HW, UART and interrupt controller) from REU address 0xfff000. Which is behind the initcramfs image and should not interfere with anything else. (Yes, this is not really a nice way to do it, but it works for now ...)
I'll add a release "v0.0.2booted" or so (see 'Releases' to the right) which will have the emulator to load from saved state plus the REU image that has Linux pre-booted in it.
Otherwise, the Linux kernel and userland is exactly the same, the only difference is that the state of the emulator has adva
\dots\par
%\par\noindent\textcolor{red}{\rule{\linewidth}{0.2mm}}
\noindent\begin{minipage}{\linewidth}
\medskip
\subsection{*@gmail.com}
\textsc{\footnotesize
{\scriptsize\faUser}\space 
About 
{\scriptsize\faCalendar}\space 
2017-01-01 
{\scriptsize\faGlobe}\space 
xkcd.com 
{\scriptsize\faThumbsOUp}\space 
\href{http://news.ycombinator.com/item?id=37333848\&utm\_term=comment}{206} 
{\scriptsize\faComments}\space 
\href{http://news.ycombinator.com/item?id=37333848\&utm\_term=comment}{18} 
}
\par\medskip\noindent
\href{https://xkcd.com/2822/?utm\_source=hackernewsletter\&utm\_medium=email\&utm\_term=fun}{
    \includegraphics[width=0.99\linewidth]{68.png}
}
\end{minipage}
\paragraph{}
\textbf{xkcd.com is best viewed with Netscape Navigator 4.0 or below on a Pentium 3±1 emulated in Javascript on an Apple IIGS at a screen resolution of 1024x1.}
\paragraph{}
 Please enable your ad blockers, disable high-heat drying, and remove your device from Airplane Mode and set it to Boat Mode. For security reasons, please leave caps lock on while browsing.
\dots\par
%\par\noindent\textcolor{red}{\rule{\linewidth}{0.2mm}}
\noindent\begin{minipage}{\linewidth}
\medskip
\subsection{Elevator Saga: An elevator programming game}
\textsc{\footnotesize
{\scriptsize\faThumbsOUp}\space 
\href{http://news.ycombinator.com/item?id=37306262\&utm\_term=comment}{126} 
{\scriptsize\faComments}\space 
\href{http://news.ycombinator.com/item?id=37306262\&utm\_term=comment}{11} 
}
\par\medskip\noindent
\href{https://play.elevatorsaga.com/index.html?utm\_source=hackernewsletter\&utm\_medium=email\&utm\_term=fun}{
    \includegraphics[width=0.99\linewidth]{69.png}
}
\end{minipage}
\paragraph{}
\textbf{Elevator Saga
The elevator programming game
Wiki \& Solutions
Documentation
Help
Your browser does not appear to support JavaScript.}
\paragraph{}
 This page contains a browser-based programming game implemented in JavaScript.
Transported
Elapsed time
Transported/s
Avg waiting time
Max waiting time
Moves
Reset
Undo reset
Apply
Save
Confused? Open the
Help and API documentation
page
Made by Magnus Wolffelt and contributors
Version
1.6.5
Source code
on GitHub
Run tests
\dots\par
%\par\noindent\textcolor{red}{\rule{\linewidth}{0.2mm}}
\noindent\begin{minipage}{\linewidth}
\medskip
\subsection{Solvethemurders.com – AI murder mystery game}
\textsc{\footnotesize
{\scriptsize\faThumbsOUp}\space 
\href{http://news.ycombinator.com/item?id=37293486\&utm\_term=comment}{4} 
{\scriptsize\faComments}\space 
\href{http://news.ycombinator.com/item?id=37293486\&utm\_term=comment}{3} 
}
\par\medskip\noindent
\href{https://solvethemurders.com?utm\_source=hackernewsletter\&utm\_medium=email\&utm\_term=fun}{
    \includegraphics[width=0.99\linewidth]{70.png}
}
\end{minipage}
\paragraph{}
\textbf{StreamlitYou need to enable JavaScript to run this app.}
\paragraph{}

\dots\par
%\par\noindent\textcolor{red}{\rule{\linewidth}{0.2mm}}
\end{multicols}

\newpage
\end{document} 